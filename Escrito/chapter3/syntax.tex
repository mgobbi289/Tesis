% Sintaxis
En el siguiente capítulo, nos dedicaremos principalmente al desarrollo del parser para nuestro intérprete.
Lo primero a realizar, será precisar formalmente la sintaxis de nuestro lenguaje.
Una vez definida, comenzaremos con la descripción de los aspectos principales sobre la implementación del parser de \Lenguaje{}.
Idealmente, esta sección del trabajo será utilizada para la documentación del futuro intérprete.

\section{Sintaxis Abstracta}

La sintaxis del lenguaje ya se expuso de manera informal en capítulos anteriores, mediante los distintos ejemplos que se fueron ilustrando.
A pesar de esto, si se quisiera hacer un estudio formal del lenguaje, lo correcto sería dar una definición precisa de la misma.
Por lo tanto, a continuación se describirá la \textit{sintaxis abstracta} de \Lenguaje{} de forma matemática.

\subsection{Expresiones}

Una expresión puede adoptar distintas formas; puede ser un valor constante, una llamada a función, una operación sobre otras expresiones o una variable con sus respectivos operadores.
Su composición se describe a continuación.

\begin{lstlisting}[style = syntax]
$\NT{expression}$ $::=$ $\NT{constant}$ | $\NT{functioncall}$ | $\NT{operation}$ | $\NT{variable}$
\end{lstlisting}

A su vez, una constante puede tomar alguno de los siguientes valores.
Los no terminales \textit{integer}, \textit{real}, \textit{bool}, y \textit{character} denotan los conjuntos de valores esperados, mientras que \textit{cname} hace referencia a los identificadores de constantes definidas por el usuario.
Los terminales \textbf{inf} y \textbf{null}, representan al infinito y al puntero nulo respectivamente.

\begin{lstlisting}[style = syntax]
$\NT{constant}$ $::=$ $\NT{integer}$ | $\NT{real}$ | $\NT{bool}$ | $\NT{character}$ | $\NT{cname}$ | $\T{inf}$ | $\T{null}$
\end{lstlisting}

Una llamada a función está compuesta por su nombre y la lista de parámetros que recibe.
La misma puede tener una cantidad arbitraria de entradas.
Notar que se utilizará la misma clase de identificadores tanto para funciones y procedimientos como para variables.

\begin{lstlisting}[style = syntax]
$\NT{functioncall}$ $::=$ $\NT{id}$ $($ $\NT{expression}$ $\ldots$ $\NT{expression}$ $)$
\end{lstlisting}

Los operadores del lenguaje están conformados por operados numéricos, booleanos, y de orden e igualdad.
Observar que será necesaria la implementación de un chequeo de tipos para asegurar el uso apropiado de los mismos.

\begin{lstlisting}[style = syntax]
$\NT{operation}$ $::=$ $\NT{expression}$ $\NT{binary}$ $\NT{expression}$ | $\NT{unary}$ $\NT{expression}$

$\NT{binary}$ $::=$ $+$ | $-$ | $*$ | $/$ | $\%$ | $||$ | $\&\&$ | $<=$ | $>=$ | $<$ | $>$ | $==$ | $!=$

$\NT{unary}$ $::=$ $-$ | $!$
\end{lstlisting}

Finalmente, describimos las variables con sus respectivos operadores.
Las mismas pueden simbolizar un único valor, un arreglo de varias dimensiones, una tupla con múltiples campos, o un puntero a otra estructura en memoria.
El no terminal \textit{fname} representa el alias de una componente para una estructura de tipo tupla definida por el usuario.

\begin{lstlisting}[style = syntax]
$\NT{variable}$ $::=$ $\NT{id}$
         | $\NT{variable}$ $[$ $\NT{expression}$ $\ldots$ $\NT{expression}$ $]$
         | $\NT{variable}$ $.$ $\NT{fname}$
         | $\star$ $\NT{variable}$
\end{lstlisting}

\subsection{Sentencias}

Las sentencias del lenguaje se dividen en las siguientes instrucciones.
La composición de la \textit{asignación} y el \textit{while} es bastante simple, por lo que se detallan también a continuación.
Notar que el no terminal \textit{sentences} se utiliza para representar la secuencia de instrucciones.

\begin{lstlisting}[style = syntax]
$\NT{sentence}$ $::=$ $\T{skip}$ | $\NT{assignment}$ | $\NT{procedurecall}$ | $\NT{if}$ | $\NT{while}$ | $\NT{for}$

$\NT{assignment}$ $::=$ $\NT{variable}$ $:=$ $\NT{expression}$

$\NT{while}$ $::=$ $\T{while}$ $\NT{expression}$ $\T{do}$ $\NT{sentences}$

$\NT{sentences}$ $::=$ $\NT{sentence}$ $\ldots$ $\NT{sentence}$
\end{lstlisting}

Para la llamada de procedimientos, se utiliza una sintaxis similar a la empleada para funciones.
Además de esto, se encuentran definidos dos métodos especiales exclusivamente para el manejo explícito de memoria del programa.

\begin{lstlisting}[style = syntax]
$\NT{procedurecall}$ $::=$ $\NT{id}$ $($ $\NT{expression}$ $\ldots$ $\NT{expression}$ $)$
             | $\T{alloc}$ $\NT{variable}$
             | $\T{free}$ $\NT{variable}$
\end{lstlisting}

La instrucción \textit{if} es bastante compleja en su composición.
Para simplificar la sintaxis abstracta, nos limitaremos a solo permitir una única opción para su especificación.
Notar que las otras versiones de esta sentencia, utilizadas en capítulos previos, se pueden obtener mediante azúcar sintáctico (\textit{syntax sugar}).

\begin{lstlisting}[style = syntax]
$\NT{if}$ $::=$ $\T{if}$ $\NT{expression}$ $\T{then}$ $\NT{sentences}$ $\T{else}$ $\NT{sentences}$
\end{lstlisting}

Finalmente, otra instrucción que presenta varias opciones es el \textit{for}.
Debido que nos encontramos desarrollando la primer versión del intérprete, algunas funcionalidades deseadas aún no se encuentran definidas.
Esto se ve reflejado en la actual instrucción.
En la misma, se pueden especificar rangos ascendentes con \textbf{to}, o descendentes con \textbf{downto}, pero la versión que admite estructuras iterables aún no ha sido determinada.

\begin{lstlisting}[style = syntax]
$\NT{for}$ $::=$ $\T{for}$ $\NT{id}$ $:=$ $\NT{expression}$ $\T{to}$ $\NT{expression}$ $\T{do}$ $\NT{sentences}$
      | $\T{for}$ $\NT{id}$ $:=$ $\NT{expression}$ $\T{downto}$ $\NT{expression}$ $\T{do}$ $\NT{sentences}$
\end{lstlisting}

\subsection{Tipos}

Los tipos que soporta \Lenguaje{} pueden dividirse en dos categorías, los nativos del lenguaje y los definidos por el usuario.
A su vez, los primeros puede separarse en básicos o estructurados.
A continuación se detallan los mismos.

\begin{lstlisting}[style = syntax]
$\NT{type}$ $::=$ $\T{int}$ | $\T{real}$ | $\T{bool}$ | $\T{char}$
      | $\NT{array}$
      | $\NT{pointer}$
      | $\NT{definedtype}$
      | $\NT{typevariable}$
\end{lstlisting}

Del lado de los tipos nativos estructurados, se tienen a los arreglos y a los punteros.
Para los primeros, hay que especificar como se definen los tamaños para sus dimensiones.
El no terminal \textit{sname} representa al tamaño polimórfico introducido en el prototipo de funciones y procedimientos.
Para los segundos, solo se debe indicar cual es el tipo de valor que se va a referenciar. 

\begin{lstlisting}[style = syntax]
$\NT{array}$ $::=$ $\T{array}$ $\NT{arraysize}$ $\ldots$ $\NT{arraysize}$ $\T{of}$ $\NT{type}$

$\NT{arraysize}$ $::=$ $\NT{natural}$ | $\NT{sname}$

$\NT{pointer}$ $::=$ $\T{pointer}$ $\NT{type}$
\end{lstlisting}

En el caso de las variables de tipo, las mismas poseen su propia clase de identificadores.
En cambio, para los tipos definidos por el usuario, además de su nombre representado por el no terminal \textit{tname}, se deben detallar los tipos en los cuales se instanciará.
Si el mismo no posee argumentos, el terminal \textbf{of} no se deberá especificar.

\begin{lstlisting}[style = syntax]
$\NT{typevariable}$ $::=$ $\NT{typeid}$

$\NT{definedtype}$ $::=$ $\NT{tname}$ $\T{of}$ $\NT{type}$ $\ldots$ $\NT{type}$
\end{lstlisting}

Cuando se declara un procedimiento, es necesario especificar el rol que cumplirá cada una de sus entradas.
Esto significa que se debe detallar, para todos los argumentos individualmente, si se emplearán para lectura (\textbf{in}), escritura (\textbf{out}), o ambas (\textbf{in/out}).

\begin{lstlisting}[style = syntax]
$\NT{io}$ $::=$ $\T{in}$ | $\T{out}$ | $\T{in/out}$
\end{lstlisting}

También existe una serie de clases predefinidas para los tipos del programa.
Las mismas representan una especie de interfaz que caracteriza las propiedades que cumplen cada uno de los tipos que las definen.
Debido que aún no se ha precisado formalmente la naturaleza de la clase \textbf{Iter}, la misma se omitirá en esta primera versión del intérprete.

\begin{lstlisting}[style = syntax]
$\NT{class}$ $::=$ $\T{Eq}$ | $\T{Ord}$
\end{lstlisting}

Finalmente, para la declaración de nuevos tipos por parte del usuario hay tres posibilidades.
Se pueden crear tipos enumerados, sinónimos de tipos y tuplas.
Para los dos últimos, se pueden especificar parámetros de tipos que permiten crear estructuras más abstractas.
Similar a lo dicho anteriormente, si los tipos declarados no poseen argumentos, entonces el terminal \textbf{of} se omite.

\begin{lstlisting}[style = syntax]
$\NT{typedecl}$ $::=$ $\T{enum}$ $\NT{tname}$ $=$ $\NT{cname}$ $\ldots$ $\NT{cname}$
         | $\T{syn}$ $\NT{tname}$ $\T{of}$ $\NT{typearguments}$ $=$ $\NT{type}$
         | $\T{tuple}$ $\NT{tname}$ $\T{of}$ $\NT{typearguments}$ $=$ $\NT{field}$ $\ldots$ $\NT{field}$

$\NT{typearguments}$ $::=$ $\NT{typevariable}$ $\ldots$ $\NT{typevariable}$

$\NT{field}$ $::=$ $\NT{fname}$ $:$ $\NT{type}$
\end{lstlisting}

\subsection{Programas}

Para finalizar con la sintaxis del lenguaje, describiremos como se especifica un programa en el mismo.
Un programa está compuesto por una serie de definiciones de tipo, seguidas de una serie de declaraciones de funciones y/o procedimientos.

\begin{lstlisting}[style = syntax]
$\NT{program}$ $::=$ $\NT{typedecl}$ $\ldots$ $\NT{typedecl}$ $\NT{funprocdecl}$ $\ldots$ $\NT{funprocdecl}$

$\NT{funprocdecl}$ $::=$ $\NT{function}$ | $\NT{procedure}$
\end{lstlisting}

A continuación, detallamos como se especifica el cuerpo de una función o procedimiento.
El mismo está conformado primero por una lista de declaraciones de variables, y segundo por una lista de instrucciones.
Para declarar una variable solo se tiene que especificar el identificador de la misma, junto con el tipo que posee.
También existe la posibilidad de definir múltiples variables en una sola declaración.

\begin{lstlisting}[style = syntax]
$\NT{body}$ $::=$ $\NT{variabledecl}$ $\ldots$ $\NT{variabledecl}$ $\NT{sentences}$

$\NT{variabledecl}$ $::=$ $\T{var}$ $\NT{id}$ $\ldots$ $\NT{id}$ $:$ $\NT{type}$
\end{lstlisting}

Una función posee un identificador propio, una lista de argumentos, un retorno, y un bloque que conforma su cuerpo.
Tanto para los argumentos, como para el retorno, solo se tienen que detallar sus identificadores junto con el tipo del valor que representarán.

\begin{lstlisting}[style = syntax]
$\NT{function}$ $::=$ $\T{fun}$ $\NT{id}$ $($ $\NT{funargument}$ $\ldots$ $\NT{funargument}$ $)$ $\T{ret}$ $\NT{funreturn}$
            $\T{where}$ $\NT{constraints}$
            $\T{in}$ $\NT{body}$

$\NT{funargument}$ $::=$ $\NT{id}$ $:$ $\NT{type}$

$\NT{funreturn}$ $::=$ $\NT{id}$ $:$ $\NT{type}$
\end{lstlisting}

Un procedimiento posee una estructura muy similar a la de una función.
Las dos diferencias fundamentales con esta, es que el primero no posee retorno ya que no producirá ningún valor como resultado, y que sus argumentos deben especificar que clase de uso se hará con los mismos.

\begin{lstlisting}[style = syntax]
$\NT{procedure}$ $::=$ $\T{proc}$ $\NT{id}$ $($ $\NT{procargument}$ $\ldots$ $\NT{procargument}$ $)$
             $\T{where}$ $\NT{constraints}$
             $\T{in}$ $\NT{body}$

$\NT{procargument}$ $::=$ $\NT{io}$ $\NT{id}$ $:$ $\NT{type}$
\end{lstlisting}

Para agregar a lo anterior, debido que los argumentos de una función o procedimiento pueden tener tipo variable, es conveniente poder imponer restricciones a los mismos.
De esta forma, se pueden crear funciones y procedimientos más abstractos que funcionen para una gran variedad de tipos, y al mismo tiempo, requerir que los mismos sean instancias de ciertas clases.

\begin{lstlisting}[style = syntax]
$\NT{constraints}$ $::=$ $\NT{constraint}$ $\ldots$ $\NT{constraint}$

$\NT{constraint}$ $::=$ $\NT{typevariable}$ $:$ $\NT{class}$ $\ldots$ $\NT{class}$
\end{lstlisting}

\section{Sintaxis Concreta}

La \textit{sintaxis concreta} de \Lenguaje{}, ya se presentó de manera informal en la introducción del lenguaje, y como para el estudio del mismo nos limitaremos a la \textit{sintaxis abstracta}, no entraremos demasiado en detalle en este aspecto.
De todas formas, consideramos importante mencionar algunas características que pueden no haber sido detalladas de forma clara en el informe.

\subsection{Identificadores}

En el lenguaje hay diversas categorías de identificadores.
Los mismos se pueden separar en dos clases, los que comienzan con minúscula (\textit{lower}), y los que comienzan con mayúscula (\textit{upper}).
Para el resto de su cuerpo, se tiene la misma estructura en ambos casos.
En la primera clase, se encuentran los identificadores de variables, funciones y procedimientos, a los tamaños de arreglos, los alias para campos de tuplas, y por último, los tipos definidos por el usuario.
Mientras, la segunda clase está conformada por los identificadores para variables de tipo, y las constantes enumeradas.

\begin{lstlisting}[style = syntax]
$\NT{id}$ $::=$ $\NT{lower}$ $\NT{rest}$

$\NT{sname}$ $::=$ $\NT{lower}$ $\NT{rest}$

$\NT{fname}$ $::=$ $\NT{lower}$ $\NT{rest}$

$\NT{tname}$ $::=$ $\NT{lower}$ $\NT{rest}$

$\NT{typeid}$ $::=$ $\NT{upper}$ $\NT{rest}$

$\NT{cname}$ $::=$ $\NT{upper}$ $\NT{rest}$
\end{lstlisting}

Con esto en mente, se puede observar que en base al contexto y al formato del identificador parseado, en la fase de \textit{análisis sintáctico} del intérprete, somos capaces de distinguir a que clase de elemento se está haciendo referencia en el código, a excepción de un caso particular.
Dentro de una expresión, no podemos diferenciar al nombre de una variable del identificador utilizado para llamar al tamaño de un arreglo.
Esto nos obliga a tener una precaución adicional a la hora de los chequeos estáticos para ser capaces de reconocer a los mismos.

Volviendo a la sintaxis, podemos ser aún más concretos.
A continuación describimos los distintos caracteres que pueden conformar un identificador del lenguaje.
Para el resto de \textit{componentes léxicos}, como los números o los caracteres literales, su estructura no presenta ninguna particularidad relevante por lo que serán omitidos.

\begin{lstlisting}[style = syntax]
$\NT{rest}$ $::=$ $($ $\NT{letter}$ | $\NT{digit}$ | $\NT{other}$ $)^{*}$

$\NT{letter}$ $::=$ $\NT{lower}$ | $\NT{upper}$

$\NT{lower}$ $::=$ a | b | ... | z

$\NT{upper}$ $::=$ A | B | ... | z

$\NT{digit}$ $::=$ 0 | 1 | ... | 9

$\NT{other}$ $::=$ $\_$ | $'$
\end{lstlisting}

\subsection{Azúcar Sintáctico}

Existen una serie de construcciones en \Lenguaje{} que son definidas mediante \textit{syntax sugar}.
Las mismas pueden ser expresadas utilizando elementos ya presentes en el lenguaje, y no expanden el poder expresivo del mismo.
Pero su utilidad radica en que ofrecen una forma sucinta y legible de especificar ciertas operaciones comunes en los algoritmos.

Una notación conveniente para acceder a los campos de una tupla señalada por un puntero es la flecha ($\rightarrow$).
De esta forma, en lugar de acceder a la memoria referenciada por un puntero con la estrella ($\star$), y luego consultar uno de los campos de la tupla con el punto ($.$), uno puede hacer uso de esta abreviatura que resulta más conveniente.
\begin{gather*}
v \rightarrow fn
\quad
\overset{def}{=}
\quad
\star v . fn
\end{gather*}

Otra notación que resulta útil en el lenguaje, es la de agrupar argumentos.
Dada la definición de una función o procedimiento, puede ocurrir que varias de sus entradas posean el mismo tipo.
En estas ocasiones es conveniente unir todas sus declaraciones en una sola.
De esta forma, queda más claro que todos los argumentos agrupados poseen el mismo tipo.
\begin{gather*}
\T{fun} \; f \; ( \; \ldots \; a_1 : \theta, a_2 : \theta, \ldots, a_n : \theta \; \ldots \; ) \; \ldots
\quad
\overset{def}{=}
\quad
\T{fun} \; f \; ( \; \ldots \; a_1, a_2, \ldots, a_n : \theta \; \ldots \; ) \; \ldots
\end{gather*}

Las últimas construcciones utilizadas para escribir código de forma concisa en el lenguaje, son todas las variantes de la sentencia \textit{if}.
Anteriormente, solo se describió una única manera para especificar la instrucción.
Con las siguientes definiciones, podemos hacer uso de los comandos introducidos en capítulos previos.
La primer notación, permite omitir el último bloque de sentencias (\textit{else}).
La segunda, nos da la capacidad de agregar una cantidad arbitraria de condicionales adicionales (\textit{elif}).
\begin{gather*}
\T{if} \; b \; \T{then} \; ss
\quad
\overset{def}{=}
\quad
\T{if} \; b \; \T{then} \; ss \; \T{else} \; \T{skip}
\\
\T{if} \; b_1 \; \T{then} \; ss_1 \; \T{elif} \; b_2 \; \T{then} \; ss_2 \; \T{else} \; ss_3
\quad
\overset{def}{=}
\quad
\T{if} \; b_1 \; \T{then} \; ss_1 \; \T{else} \; \T{if} \; b_2 \; \T{then} \; ss_2 \; \T{else} \; ss_3
\end{gather*}