% Parser
\section{Implementación del Parser}

Habiendo presentado de manera informal la sintaxis concreta y definido de manera formal la sintaxis abstracta, estamos en condiciones para describir los detalles principales sobre el desarrollo del parser para \Lenguaje{}.
Haremos mención de las decisiones más relevantes tomadas durante su implementación, las dificultades encontradas en el camino, y algunas limitaciones que debieron ser resueltas.

\subsection{Librerías}

Inicialmente se comenzó utilizando la librería \Parsec{}~\cite{Parsec}.
La decisión se tomó debido que algunos de nosotros ya estábamos familiarizados con su uso por proyectos anteriores.
Más adelante, en las etapas finales del desarrollo del parser, se decidió migrar el código a \Megaparsec{}~\cite{Megaparsec}.
La transición fue justificada por las limitaciones que presentaba la primera opción frente a la segunda, que además de solucionar algunas de las dificultades de la implementación de forma sencilla, ofrece un rango de funcionalidades más diverso que puede beneficiar al desarrollo futuro del intérprete.

\subsubsection{Parsec}

\Parsec{} es una librería para el diseño de un parser monádico, implementada en \Haskell{}, y escrita por \textit{Daan Leijen}.
Es simple, segura, rápida, y posee buena documentación.
Para la etapa inicial de desarrollo, resultó ser una herramienta intuitiva y fácil de manejar.
A pesar de sus cualidades, para este proyecto en particular, la librería no ofrecía la suficiente flexibilidad y funcionalidad que buscábamos.
Una primera versión completa del parser fue desarrollada utilizando esta librería.
% IDEA: La totalidad del parser (al menos en esta primera versión del intérprete) fue implementada usando la librería mencionada.

\subsubsection{Megaparsec}

\Megaparsec{} se puede considerar el sucesor extraoficial de \Parsec{}, escrita por \textit{Mark Karpov}.
Partiendo de las bases definidas por esta última, la librería busca ofrecer mayor flexibilidad para la configuración del parser y una generación de mensajes de error más sofisticada respecto a su antecesor.
La herramienta resulta familiar para todo el que tenga ciertos conocimientos básicos sobre \Parsec{}.
La transición de librerías se vio aliviada por esta característica, debido a la semejanza entre ambas.

\subsubsection{Breve Comparación}

Como mencionamos previamente, debido que \Parsec{} no resultó ser la herramienta ideal para el desarrollo de nuestro parser, se decidió comenzar a utilizar \Megaparsec{}.
Las razones puntuales que nos llevaron a tomar esta decisión se listan a continuación.
\begin{itemize}
    \item \underline{Análisis Léxico:}
    Ambas librerías implementan un mecanismo sencillo para definir un \textit{analizador léxico}, dentro del mismo parser.
    De esta forma, se simplifica el diseño del parser al unificar estas dos fases fuertemente acopladas.
    La diferencia entre ambas, radica en el hecho que \Parsec{} es demasiado inflexible en este aspecto.
    Para ciertas cuestiones, se tuvo que redefinir gran parte de la implementación de la librería para poder acomodarla a nuestras necesidades.
    En cambio, \Megaparsec{} no impone ninguna estructura sobre el \textit{analizador léxico}, y solo provee funcionalidades básicas elementales para su definición.
    \item \underline{Mensajes de Error:}
    La generación de mensajes de error en el análisis sintáctico (al igual que para los chequeos estáticos) es una tarea sumamente importante.
    La primera herramienta utilizada, ofrecía una forma simple y concisa para el informe de errores.
    En la misma, se podían especificar cuales eran los \textit{tokens} esperados (o inesperados) por el parser al momento de fallar, junto con el mensaje informativo asociado a esta.
    A pesar de esto, la segunda opción presenta una generación de errores mucho más desarrollada.
    Además de conservar las funcionalidades previas, se pueden configurar nuevas clases de errores junto con la forma que los mensajes de error son presentados.
    \item \underline{Desarrollo Futuro:}
    Una vez que el desarrollo del parser se encuentre lo suficientemente avanzado, será necesario volver a esta etapa y adecuarla a las nuevas necesidades que hayan surgido en el camino.
    En este aspecto, \Megaparsec{} ofrece una serie de funcionalidades adicionales que no se encuentran en \Parsec{}.
    Una de ellas es el soporte para múltiples errores, junto con la capacidad de atrapar errores, lo que permite un control mucho más amplio sobre la información que recibe el usuario al analizar su código; las cuales se logran gracias a la recuperación de errores de parseo.
    También existe la posibilidad de agregar casos de test para aumentar la certeza que el funcionamiento del parser implementado es correcto.
    Una última característica que puede resultar útil en un futuro, es el parseo \textit{sensible a la indentación} que ofrece la librería.
\end{itemize}

\subsection{Información de Posición}

Una de las tareas importantes que debe realizar el parser es la generación de mensajes de error informativos y precisos con el fin de facilitar su corrección. Es fundamental entonces poder indicar exactamente  \textit{donde} ocurre el error en el programa.
Cuando se produce un error durante la etapa de parseo, la librería \Megaparsec{} ya incorpora un buen mecanismo para señalar la posición en donde se ha encontrado la falla.
De todas maneras la implementación de los chequeos estáticos del lenguaje, que presentaremos en el siguiente capítulo, también debe poder informar de forma precisa los errores producidos ahora en esta etapa.
Para solucionar esto, se implementó la sintaxis abstracta utilizando un método similar al empleado por el compilador de \Haskell{}~\cite{Haskell}, \texttt{GHC}~\cite{GHC}.
Informalmente la estrategia es encapsular a toda construcción sintáctica que consideremos relevante para la generación de errores con la información de su posición correspondiente en el programa declarado.

El módulo \lstinline[style = module]{Syntax.Located} define al \textit{tipo de datos} \lstinline[style = haskell]{Located} que almacena las lineas y columnas de inicio y fin de un determinado elemento, junto con el correspondiente elemento; actualmente el lenguaje no posee un sistema de módulos por lo que la información de lineas y columnas es suficiente, ya que nunca se parsea ni chequea estáticamente más de un archivo.

\begin{lstlisting}[ style = haskell, caption = Información de Posición en Parser, label = Located ]
-- Parsing Information
data Located e = L { info :: Info
                   , item :: e
                   }

-- Position Information
data Info = I { sLin :: Line
              , sCol :: Column
              , eLin :: Line
              , eCol :: Column
              }
\end{lstlisting}

A medida que avanza el parser, inmediatamente después de parsear algún elemento, este se encapsula con la información de su posición y se almacena en el \textit{árbol de sintaxis abstracta}.
Por ejemplo en la implementación de \lstinline[style = module]{Syntax.Expr} para las expresiones del lenguaje, podemos almacenar tanto la posición de una expresión particular como todas las de sus subexpresiones.
Lo cual tiene como ventaja que en el caso de encontrarse una falla (por ejemplo, un error de tipos) se pueda exhibir toda la \textit{traza} de parseo junto con sus posiciones correspondientes.
Notar que en la definición del tipo \lstinline[style = haskell]{Expr} no existen subexpresiones con este mismo tipo, sino que en su lugar tienen tipo \lstinline[style = haskell]{LExpr}, es decir expresiones encapsuladas con la información sobre su posición.
Por lo general todo nombre de tipo prefijado con {\ttfamily \color{Cyan} L} significará que este está encapsulado.

\begin{lstlisting} [ style = haskell, caption = Expresiones del Lenguaje, label = Expr ]
-- Expressions
data Expr = Const LConstant
          | Loc LLocation
          | UOp UnOp LExpr
          | BOp BinOp LExpr LExpr
          | FCall Id [LExpr]

type LExpr = Located Expr
\end{lstlisting}

\subsection{Módulos}

A continuación describiremos brevemente los distintos módulos en los que se ha dividido la implementación del parser.
Haremos mención de los detalles más relevantes de cada uno de estos elementos, junto con el propósito de los mismos.

En \lstinline[style = module]{Parser.Position} se proveen todas las funcionalidades necesarias para poder calcular la ubicación en el archivo del elemento que se intenta parsear.
% IDEA: Es utilizado en la mayoría de los módulos para el \textit{análisis sintáctico} del intérprete.
Como mencionamos previamente, esta tarea es fundamental para luego poder dar mensajes de errores precisos e informativos al usuario.
Haciendo uso de la función \lstinline[style = haskell]{getSourcePos}, provista por la librería, podemos extraer la posición actual del parser y luego, ligarla con el elemento sintáctico correspondiente.

El módulo \lstinline[style = module]{Parser.Lexer} es uno de los más importantes, ya que implementa la totalidad del \textit{analizador léxico} del lenguaje.
En este se implementan funciones para parsear todas las clases de identificadores, los valores constantes (como los numéricos, por ejemplo), y las \textit{palabras claves} y operadores del lenguaje.
Inicialmente cuando se utilizaba la librería \Parsec{}, se tuvo que redefinir la mayoría de las funciones que implementaba debido que las mismas consumían automáticamente todos los \textit{whitespaces} (espacios en blanco, comentarios, saltos de línea, etc.) al parsear un elemento.
Lo cual impedía poder calcular de manera precisa la posición de las distintas estructuras sintácticas del lenguaje.
Luego de la transición, debido que \Megaparsec{} delega la responsabilidad del consumo de \textit{whitespace} al usuario, se pudo hacer uso de las funciones auxiliares que brinda la librería, y se simplificó el módulo.

En \lstinline[style = module]{Parser.Expr} se parsean las diversas expresiones del lenguaje.
Una particularidad interesante de este módulo, es el uso de la función \lstinline[style = haskell]{makeExprParser}.
Dado un parser de términos, que serían los elementos básicos que conforman una expresión, y una tabla de operadores, donde se debe especificar la asociatividad y precedencia de cada uno, la función construye un parser para expresiones basado en los mismos.
En nuestro caso, se hizo uso de este mecanismo para las expresiones y las variables del lenguaje.
Para el primero, su implementación es directa debido que es una situación usual.
En cambio, para el segundo, se tuvo que interpretar a las distintas operaciones para el acceso de variables como operadores de expresiones para poder aprovechar la función especificada en la librería.

El módulo \lstinline[style = module]{Parser.Statement} se encarga de obtener las sentencias especificadas en el código.
A diferencia de la \textit{sintaxis abstracta}, en la implementación del lenguaje se permiten múltiples formas para detallar la instrucción condicional \textit{if}.
Para la misma, el componente \textit{else} es opcional, y además, se pueden agregar una cantidad arbitraria de condicionales \textit{elif}.
Otra particularidad, es la forma de obtener la ubicación para la asignación.
Debido que esta es la única sentencia que no posee un delimitador final, calcular su posición no es una tarea inmediata.

En \lstinline[style = module]{Parser.Decl} se parsean todas las declaraciones del lenguaje.
Por un lado se tienen a las definiciones de tipo, las cuales abarcan a las enumeraciones, los sinónimos, y las tuplas.
Por el otro tenemos a las definiciones de funciones y procedimientos, junto con todos los elementos que las conforman, como sus parámetros y restricciones.

Los módulos restantes no presentan complejidad adicional en comparación con los descriptos recientemente.
El módulo \lstinline[style = module]{Parser.Type} obtiene los distintos tipos sintácticamente válidos del lenguaje.
En \lstinline[style = module]{Parser.Class} se parsean todas las clases predefinidas en el lenguaje.
Y finalmente, el parser para programas se implementa en \lstinline[style = module]{Parser.Program}.