% Parser
\section{Parser}

Ya nos encontramos en condiciones para describir los detalles principales sobre el desarrollo del parser para \Lenguaje{}.
Haremos mención de las decisiones más relevantes tomadas durante la implementación del mismo, las dificultades encontradas en el camino, y algunas limitaciones que debimos resolver.

\subsection{Librerías}

Inicialmente, se comenzó utilizando la librería \Parsec{}~\cite{Parsec}.
Esta decisión se tomó debido que algunos de nosotros ya estábamos familiarizados con su uso por proyectos anteriores.
Más adelante, en las etapas finales del desarrollo del parser, se decidió migrar el código a \Megaparsec{}~\cite{Megaparsec}.
Esta transición fue justificada por las limitaciones que presentaba la primera opción frente a la segunda, que además de solucionar algunas de las dificultades de la implementación de forma sencilla, ofrece un rango de funcionalidades más diverso que puede beneficiar al desarrollo futuro del intérprete.

\subsubsection{Parsec}

\Parsec{} es una librería para el diseño de un parser monádico, implementada en \Haskell{}, y escrita por \textit{Daan Leijen}.
Es simple, segura, rápida, y posee buena documentación.
Para la etapa inicial de desarrollo, resultó ser una herramienta intuitiva y fácil de manejar.
A pesar de esto, para este proyecto en particular, la misma no ofrecía la suficiente flexibilidad y funcionalidad que buscábamos.
La totalidad del parser (al menos en esta primera versión del intérprete) fue implementada usando esta librería.

\subsubsection{Megaparsec}

\Megaparsec{} se puede considerar el sucesor extraoficial de \Parsec{}, escrita por \textit{Mark Karpov}.
Partiendo de las bases definidas por esta última, la librería busca ofrecer mayor flexibilidad para la configuración del parser y una generación de mensajes de error más sofisticada respecto a su antecesor.
La herramienta resulta familiar para todo el que tenga ciertos conocimientos básicos sobre \Parsec{}.
La transición de librerías se vio aliviada por esta característica, debido a la semejanza entre ambas.

\subsubsection{Comparación}

Como mencionamos previamente, debido que \Parsec{} no resultó ser la herramienta ideal para el desarrollo de nuestro intérprete, se decidió comenzar a utilizar \Megaparsec{}.
Las razones puntuales que nos llevaron a tomar esta decisión se listan a continuación.
\begin{itemize}
    \item \underline{Análisis Léxico:}
    Ambas librerías implementan un mecanismo sencillo para definir un \textit{analizador léxico}, dentro del mismo parser.
    De esta forma, se simplifica el diseño del intérprete al unificar estas dos fases fuertemente acopladas.
    La diferencia entre ambas, radica en el hecho que \Parsec{} es demasiado inflexible en este aspecto.
    Para ciertas cuestiones, se tuvo que redefinir gran parte de la implementación de la librería para poder acomodarla a nuestras necesidades.
    En cambio, \Megaparsec{} no impone ninguna estructura sobre el \textit{analizador léxico}, y solo provee funcionalidades básicas elementales para su definición.
    \item \underline{Mensajes de Error:}
    La generación de mensajes de error en el análisis sintáctico (al igual que en el análisis semántico) es una tarea sumamente importante para el intérprete.
    La primera herramienta utilizada, ofrecía una forma simple y concisa para el informe de errores.
    En la misma, se podían especificar cuales eran los \textit{tokens} esperados (o inesperados) por el parser al momento de fallar, junto con el mensaje informativo asociado a esta.
    A pesar de esto, la segunda opción presenta una generación de errores mucho más desarrollada.
    Además de conservar las funcionalidades previas, se pueden configurar nuevas clases de errores junto con la forma que los mensajes de error son presentados al usuario.
    \item \underline{Desarrollo Futuro:}
    Una vez que el desarrollo del intérprete se encuentre lo suficientemente avanzado, será necesario volver a esta etapa y adecuarla a las nuevas necesidades que hayan surgido en el camino.
    En este aspecto, \Megaparsec{} ofrece una serie de funcionalidades adicionales que no se encuentran en \Parsec{}.
    Una de ellas es el soporte para múltiples errores, junto con la capacidad de atrapar errores, lo que permite un control mucho más amplio sobre la información que recibe el usuario al analizar su código.
    También existe la posibilidad de agregar casos de test para aumentar la certeza que el funcionamiento del parser implementado es correcto.
    Una última característica que puede resultar útil en un futuro, es el parseo \textit{sensible a la indentación} que ofrece la librería.
\end{itemize}

\subsection{Información de Posición}

Una tarea fundamental que debe realizar cualquier compilador, o en nuestro caso intérprete, es informar al usuario sobre los errores sintácticos o semánticos que se hayan detectado durante el análisis y procesamiento de su programa.
Debido a esto, la generación de mensajes de error informativos y precisos es una cualidad deseada en esta clase de herramientas, ya que facilitan la corrección de los mismos por parte del programador.
Una propiedad que se puede deducir de lo anterior, es que para que un mensaje de error sea adecuado es fundamental que el mismo pueda indicar puntualmente \textit{donde} ocurre este error en el programa.

En este ciclo inicial de desarrollo del intérprete, tanto la fase de \textit{análisis sintáctico} como la de \textit{análisis semántico} podrán encontrar fallas en el código, y deberán informar sobre el problema detectado al usuario.
Para la primer etapa, cuando se produce un error durante el parsing de algún elemento sintáctico, la librería \Megaparsec{} ya incorpora un mecanismo para señalar la posición en el archivo donde se ha encontrado la falla.
Haciendo uso de la misma, conseguimos tener una generación de mensajes de error claros para la fase de \textit{análisis sintáctico}.
Cuando avanzamos a etapas posteriores en el intérprete, necesitamos alguna forma de poder vincular las fallas detectadas durante las mismas con la ubicación en el archivo involucrado en el error.

Para solucionar este problema, se adaptó la sintaxis de una forma similar a como lo hace el compilador de \Haskell{}, \texttt{GHC}~\cite{GHC}.
Con esto nos referimos a que todo elemento sintáctico del lenguaje que puede ser relevante en la generación de errores, es envuelto con la información de posición correspondiente a su ocurrencia en el código.
De esta forma, se definió en el módulo \lstinline[style = module]{Syntax.Located}, el siguiente \textit{datatype}~(\ref{Located}).
Recordar que en esta primera versión del intérprete solo trabajamos con programas definidos en un único módulo, por lo que almacenar solamente las líneas y columnas de inicio y fin de un determinado elemento, es suficiente para generar un mensaje de error preciso.

\begin{lstlisting}[ style = haskell, caption = Información de Posición en Parser, label = Located ]
-- Parsing Information
data Located e = L { info :: Info
                   , item :: e
                   }

-- Position Information
data Info = I { sLin :: Line
              , sCol :: Column
              , eLin :: Line
              , eCol :: Column
              }
\end{lstlisting}

A medida que avanza el parser en el análisis de un archivo, cuando se obtiene un elemento sintáctico (por ejemplo, una expresión), se encapsula el mismo con la información de su posición y se almacena en el \textit{árbol de sintaxis abstracta} que se genera como representación intermedia del código.
Siendo un poco más puntuales, podemos ver el ejemplo de las expresiones del lenguaje.
En \lstinline[style = module]{Syntax.Expr} se especifican las mismas de la siguiente forma~(\ref{Expr}).
Con esta definición, podemos almacenar tanto la posición de una expresión particular como todas las de sus subexpresiones.
Esto tiene como ventaja que en el caso de encontrarse una falla (por ejemplo, un error de tipos) en el código, se puede exhibir toda la \textit{traza} de análisis junto con sus posiciones correspondientes. 

\begin{lstlisting} [ style = haskell, caption = Expresiones del Lenguaje, label = Expr ]
-- Expressions
data Expr = Const LConstant
          | Loc LLocation
          | UOp UnOp LExpr
          | BOp BinOp LExpr LExpr
          | FCall Id [LExpr]

type LExpr = Located Expr
\end{lstlisting}

\subsection{Módulos}

A continuación, describiremos brevemente los distintos módulos en los que se divide la implementación del parser.
Mencionaremos los detalles más relevantes de cada uno de estos elementos, junto con el propósito de los mismos.

En \lstinline[style = module]{Parser.Position} se proveen todas las funcionalidades necesarias para poder calcular la ubicación en el archivo del elemento que se intenta parsear.
Es utilizado en la mayoría de los módulos para el \textit{análisis sintáctico} del intérprete.
Como mencionamos previamente, esta tarea es fundamental para luego poder dar mensajes de errores precisos e informativos al usuario.
Haciendo uso de la función \lstinline[style = haskell]{getSourcePos}, provista por la librería, podemos extraer la posición actual del parser y luego, ligarla con el elemento sintáctico correspondiente.

El módulo \lstinline[style = module]{Parser.Lexer} es uno de los más importantes del código.
Comprende la totalidad del \textit{analizador léxico} del lenguaje.
En el mismo, se implementan funciones para parsear todas las clases de identificadores, los valores constantes (como los numéricos, por ejemplo), y las \textit{palabras claves} y operadores de \Lenguaje{}.
Inicialmente, cuando se utilizaba la librería \Parsec{}, se tuvo que redefinir la mayoría de las funciones que implementaba debido que las mismas consumían automáticamente todos los \textit{whitespaces} (espacios en blanco, comentarios, saltos de línea, etc.) al parsear un elemento.
Esto impedía poder calcular de forma precisa la posición de las distintas estructuras sintácticas del lenguaje.
Luego de la transición, debido que \Megaparsec{} delega la responsabilidad del consumo de \textit{whitespace} al usuario, se pudo hacer uso de las funciones auxiliares que brinda la librería, y se simplificó el módulo.

En \lstinline[style = module]{Parser.Expr} se parsean las diversas expresiones del lenguaje.
Una particularidad interesante de este módulo, es el uso de la función \lstinline[style = haskell]{makeExprParser}.
Dado un parser de términos, que serían los elementos básicos que conforman una expresión, y una tabla de operadores, donde se debe especificar la asociatividad y precedencia de cada uno, la función construye un parser para expresiones basado en los mismos.
En nuestro caso, se hizo uso de este mecanismo para las expresiones y las variables del lenguaje.
Para el primero, su implementación es directa debido que es una situación estándar.
En cambio, para el segundo, se tuvo que interpretar a las distintas operaciones para el acceso de variables como operadores de expresiones para poder aprovechar la función especificada en la librería.

El módulo \lstinline[style = module]{Parser.Statement} se encarga de obtener las sentencias especificadas en el código.
A diferencia de la \textit{sintaxis abstracta}, en la implementación del lenguaje se permiten múltiples formas para detallar la instrucción condicional \textit{if}.
Para la misma, el componente \textit{else} es opcional, y además, se pueden agregar una cantidad arbitraria de condicionales \textit{elif}.
Otra particularidad, es la forma de obtener la ubicación para la asignación.
Debido que esta es la única sentencia que no posee un delimitador final, calcular su posición no es una tarea inmediata.

En \lstinline[style = module]{Parser.Decl} se parsean todas las declaraciones del lenguaje.
Esto involucra diversas construcciones de distinto índole.
Se obtienen las definiciones de tipo, los cuales abarcan a las tuplas, los sinónimos y las enumeraciones.
También se especifican las declaraciones de instancias de clases para los mismos.
Además, se parsean las definiciones de funciones y procedimientos, junto con todos los elementos que las conforman, como sus argumentos y restricciones.

Los últimos archivos no presentan ninguna complejidad adicional.
El módulo \lstinline[style = module]{Parser.Type} obtiene los distintos tipos admitidos en el lenguaje.
En \lstinline[style = module]{Parser.Class} se parsean todas las clases predefinidas en el mismo.
Y finalmente, \lstinline[style = module]{Parser.Program} acepta los programas, sintácticamente válidos, de \Lenguaje{}.

%Diferencias entre Librerías
\iffalse
Megaparsec vs Parsec
Since Megaparsec is a fork of Parsec, we are bound to list the main differences between the two libraries:
  
  Better error messages. Megaparsec has typed error messages and custom error messages, it can also report multiple parse errors at once.
  
  Megaparsec can show the line on which parse error happened as part of parse error. This makes it a lot easier to figure out where the error happened.
  
  Some quirks and bugs of Parsec are fixed.
  
  Better support for Unicode parsing in Text.Megaparsec.Char.
  
  Megaparsec has more powerful combinators and can parse languages where indentation matters out-of-the-box.
  
  Better documentation.
  
  Megaparsec can recover from parse errors “on the fly” and continue parsing.
  
  Megaparsec allows us to conditionally process parse errors inside your parser before parsing is finished. In particular, it's possible to define regions in which parse errors, should they happen, will get a “context tag”, e.g. we could build a context stack like “in function definition foo”, “in expression x”, etc.
  
  Megaparsec is faster and supports efficient operations tokens, takeWhileP, takeWhile1P, takeP, like Attoparsec
\fi

% Otras secciones
\iffalse

\subsection{Monada Empleada}
type Parser = ...

\subsection{Decisiones Importantes}
- Paréntesis en Tipos Definidos
- Paréntesis en expresiones contra variables

\fi

% Ideas
\iffalse
% Diferencia de Sintaxis
por supuesto comentando que no está tan alejada de la sintaxis concreta; que para el estudio teórico del lenguaje no nos interesa
% Lexer
Una subseccion que se me ocurre es la del Lexer.hs
...
donde estan la mayoria de la toma de decisiones y cosas como: como se especifica un identificador o poruqe no se consumen whitespace de forma automatica
...
Yo solo hablo de parsec y de lo que implemente, no? O sea, megaparsec ni lo menciono? Tambien tendria que aclarar que una expansion del lenguaje (que no hice / hago) es poder definir instancias para los tipos definidos por el usuario y la parte de modulos
% Modulos
algo que podes hacer seguro es una pequeña reseña de que tiene cada módulo
del parser digo
o sea, puede ser una oración para cada módulo
% Megaparsec
- de Parsec a Megaparsec ; creo estaría bueno comentar un poquito si te sale
...
pero bueno, igual yo te digo porque me pareció mejor. Primero no tiene el makeTokenParser que nos traía problemas y nos limitaba; por eso en parte tuviste que redefinir ciertas cosas. Otra cosa piola es que tiene soporte para múltiples errores (capaz llego a tener algo implementado) y por último y no menos importante el informe de errores está mucho más desarrollado
- Tanto Parsec como Megaparsec (cambia mínimamente) tienen el buildExpressionParser. Acá podemos comentar un poquito de que trata y contar que lo usaste para parsear el *, [3] y .field pensándolos como operadores.
ah, seguramente esta bueno dejar claro en una pequeña introducción a la sección que definición del tipo de parser usas, nosotros si no recuerdo mal estamos usando esta
+ Let us define a type synonym (typically called Parser) like this:
...
type Parser = Parsec Void Text
-- Custom error component Type of input stream
% Resumen
En resumen, la sección puede arrancar contando las librerías que usaste (Parsec y Megaparsec) con algún poquito de contraste entre ellas si quieres(Megaparsec tiene mejor informe de errores, tiene funciones para implementar multi-errores y no es tan restrictivo con el tema del tokenizador).
Después la definición del tipo Parser (si es que te hace falta para después).
Después detallas los módulos que conforman el parser y en una oración de qué tratan cada uno.
Finalmente podes contar algo sobre la función makeExprParser (así se llama en Megaparsec, en realidad así se llama en un módulo más general de la librería parser-combinators de la cual Megaparsec define una instancia para algún tipo) esto inicialmente puede ser como la usaste de manera esperable para los operadores aritméticos, etc. y después bien importante como la usaste para el de operadores sobre punteros, arrays y tuplas.
\fi