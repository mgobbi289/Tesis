% Introducción

En el siguiente proyecto desarrollaremos la semántica estática de un lenguaje de programación inspirado en el pseudocódigo utilizado por la materia \Materia{}.
A diferencia de la mayoría de los lenguajes de programación conocidos, en el trabajo se presenta un desarrollo formal del lenguaje previo a su implementación.
Se acompañará el avance del proyecto con el diseño de un intérprete adecuado.
En el escrito nos concentraremos principalmente en el estudio teórico del lenguaje, donde solo se comentarán algunos detalles sobre su implementación.
% IDEA: Partiendo de un estudio teórico de sus propiedades, se dará una definición formal del mismo que luego tendrá como resultado su implementación.
% IDEA: Con el avance del proyecto, se acompañará al desarrollo con el diseño de un intérprete adecuado.
A lo largo del trabajo se empleará el nombre tentativo \Lenguaje{} para hacer referencia al lenguaje.
% IDEA: Un nombre tentativo para este lenguaje es \Lenguaje{}, el cual será empleado a lo largo del trabajo.

\section{Motivación}

En la materia \Materia{}, de \Facultad{}, se utiliza desde hace años un pseudocódigo para la enseñanza de conceptos tales como análisis de algoritmos, definición de tipos abstractos de datos, o la comprensión de distintas técnicas de programación.
Este pseudocódigo está inspirado en el lenguaje imperativo \Pascal{}, diseñado por \textit{Niklaus Wirth} cerca de 1970~\cite{Pascal}, al cual se le han ido realizando modificaciones de acuerdo a las necesidades didácticas de la materia.
Algunas características que se han mantenido comprenden la sintaxis verbosa, el tipado fuerte para expresiones, y el formato estructurado del código.
% ... lenguaje Pascal [contás más características, por ejemplo como que es imperativo], al cual ...
Haciendo un análisis exhaustivo de las clases y los ejemplos en los que se utiliza el lenguaje en la materia, observamos que su definición aún no está completamente clara, y que el grado de precisión con el que se declaran los programas varía entre una presentación y otra.

\begin{lstlisting}[ style = pseudo, caption = Permutación de Valores, label = pseudoSwap ]
{ PRE: 1 <= i, j <= n }
proc swap ( in / out a : array [1..n] of T, in i, j : nat )
  var temp : T
  temp := a[i]
  a[i] := a[j]
  a[j] := temp
end proc
\end{lstlisting}

Mediante los procedimientos \lstinline[style = pseudo]{swap} y \lstinline[style = pseudo]{selectionSort}, declarados en el pseudocódigo de la materia, se define un algoritmo para ordenar arreglos de menor a mayor.
Cada uno de los pasos ha efectuar por el algoritmo es detallado con precisión, por lo que es sencillo imaginar una posible ejecución del programa dado con un arreglo \lstinline[style = pseudo]{a} cualquiera.
La traducción del código a un lenguaje de programación imperativo, como por ejemplo \C{}, resulta ser casi directa.
% IDEA: No hay ambigüedades, por lo que es intuitivo saber que hacer en cada caso

\begin{lstlisting}[ style = pseudo, caption = Ordenación por Selección, label = pseudoSort ]
proc selectionSort ( in/out a : array [1..n] of T )
  var minP : nat
  for i := 1 to n do
    minP := i
    for j := i + 1 to n do
      if a[j] < a[minP] then
        minP := j
      fi
    od
    swap(a, i, minP)
  od
end proc
\end{lstlisting}

Con la función \lstinline[style = pseudo]{cambio} se busca resolver el \textit{problema de la moneda}, donde se intenta pagar un monto de dinero \lstinline[style = pseudo]{m} con la menor cantidad de monedas posibles cuyas denominaciones válidas se encuentran en el conjunto \lstinline[style = pseudo]{C}; el cual tiene una cantidad infinita de cada una.
A diferencia del ejemplo previo ciertos pasos del algoritmo resultan ambiguos, donde en el pseudocódigo se permite escribir una oración en \textit{lenguaje natural} para describir la acción a realizar durante su ejecución.
Un problema que conlleva esto para un programador no experimentado es responder a la pregunta \textit{¿Cuál es el grado de ambigüedad permitido?}
% IDEA: Claramente en esta clase de problemas se trabaja con un grado de abstracción elevado.

% FIX de comentarios!
\begin{lstlisting}[ style = pseudo, caption = Problema de la Moneda, label = pseudoCoin, otherkeywords = { \{, \}, +, -, <, >, :, :=, (, ), \, } ]
fun cambio ( m : $monto$, C : $Conjunto \; de \; Monedas$ )
ret S : $Conjunto \; de \; Monedas$
  var c, resto : $monto$
  S := {} #\ttfamily \itshape \color{OliveGreen} \{ Inicializamos S con el conjunto vacío \}#
  resto := m
  while resto > 0 do
    c := $mayor \; elemento \; de$ C
    C := C - {c}
    S := S $\color{RoyalBlue} \cup$ { resto div c $monedas \; de \; denominaci \acute{o} n$ c }
    resto := resto mod c
  od
end fun
\end{lstlisting}

% [Acá pondría ejemplos de eso. Algunos donde el lenguaje sea bien preciso, y otros donde haya un grado de abstracción e imprecisión grande].

En este trabajo nos proponemos definir con precisión un lenguaje como el que se utiliza en la materia, que permita al estudiante poder implementar y ejecutar los algoritmos presentados en el teórico y en el práctico.
Además del beneficio desde el punto de vista didáctico, la definición formal e implementación de un lenguaje con estas características reviste interés propio.
En el mismo se utilizan algunos conceptos avanzados de lenguajes de programación como el polimorfismo paramétrico, una variante primitiva de polimorfismo \textit{ad hoc}, el manejo de memoria dinámica mediante punteros, pasaje por valor y referencia a través de funciones y procedimientos, entre otros.
% [Quizás se puede presentar mejor esas características interesantes del lenguaje].
Para lograr el objetivo propuesto debemos primero definir precisamente la sintaxis del lenguaje, identificar las ambigüedades que naturalmente tiene debido al uso didáctico con el que fue creado, luego definir chequeos estáticos como el tipado, el alcance de variables, atributos sobre parámetros de funciones y procedimientos, entre otros.
En una segunda etapa, que excede esta tesis, se definirá formalmente la semántica dinámica y un intérprete interactivo que permita ejecutar algoritmos observando el estado paso a paso.

\section{Conceptos}

A continuación introduciremos algunos conceptos generales que muy probablemente aparecen en el estudio y definición formal de cualquier lenguaje de programación, y por lo tanto también en este trabajo puntual.
En particular nos enfocaremos en la sintaxis, concreta y abstracta, y los chequeos estáticos.

La \textbf{sintaxis concreta} define que cadenas de caracteres serán consideradas programas del lenguaje, es decir los programas que uno finalmente escribe son caracterizados por esta gramática, y por lo general consiste de un conjunto de reglas, también llamadas producciones, que definen la manera de declarar programas de acuerdo a una \textit{gramática libre de contexto} del lenguaje.
Notar que si esta gramática define como mencionamos qué programas están correctamente escritos, entonces es porque contempla detalles precisos de la escritura de las construcciones del lenguaje como los caracteres válidos para nombrar variables, las llaves para delimitar bloques de código, o las palabras claves correspondientes para denotar determinadas construcciones.
Si por ejemplo consideramos al lenguaje de programación \C{}, podríamos definir un pequeño fragmento de su sintaxis concreta para la declaración de funciones.
\begin{gather*}
\NT{concreteFun} ::= \NT{type} \; \NT{id} \; ( \; [ \; \NT{id} : \NT{type} \; ]^* \; ) \; \{ \; \NT{body} \; \}
\end{gather*}

Asumiendo tenemos definidas las gramáticas para el resto de las construcciones no terminales $\NT{type}$, $\NT{id}$, y $\NT{body}$, notar que escribir una función en \C{} consiste en escribir el tipo de retorno, su nombre seguido de los parámetros que necesariamente deben estar encerrados entre paréntesis y separados por comas, y finalmente entre llaves escribimos el cuerpo de la función.
Por ejemplo, un fragmento de la implementación de \texttt{factorial} podría ser de la siguiente forma.
\begin{gather*}
\texttt{int} \; \texttt{factorial} \; ( \texttt{n} : \texttt{int} ) \; \{ \; body_{conc} \; \}
\end{gather*}

Todos estos detalles burocráticos de escritura son importantes ya que, si bien algunas veces la sobrecargan, ayudan entre otras cosas a la lectura y organización del código.
Sin embargo, una representación más abstracta de las componentes sintácticas que definen un programa suele resultar más conveniente para el estudio formal de propiedades como el chequeo de tipos, entre otras.
Así como también una representación interna de un programa en la implementación de un compilador o intérprete.
% IDEA: Se utiliza un \textit{árbol de sintaxis concreta} para representar la derivación de un programa a través de las reglas de la respectiva sintaxis.
% IDEA: Un intérprete toma un fragmento de texto, el cual debe respetar la definición de las reglas sintácticas, y lo convierte en una representación intermedia del programa.
% IDEA: El proceso de transformación es llevado a cabo por las etapas de \textit{análisis léxico} y \textit{análisis sintáctico} del intérprete.
% IDEA: Notar que si se quisiera hacer un estudio del lenguaje, utilizar su sintaxis concreta puede agregar complicaciones innecesarias.
% IDEA: Los programas del lenguaje no son simplemente cadenas de caracteres, sino entidades abstractas representadas por cadenas de caracteres.
% IDEA: Donde la gramática declarada solo captura detalles sobre la estructura concreta de las frases del lenguaje.

La \textbf{sintaxis abstracta} define de manera independiente de cualquier representación particular a las \textit{frases abstractas} que conforman a un lenguaje.
Definiendo una \textit{gramática abstracta} nos enfocamos solo en la información que nos resulta importante de una frase, omitiendo por lo tanto detalles de su representación específica.
Una posible representación abstracta de las funciones de \C{} es la siguiente, donde la declaración de una función (en su versión abstracta) tendrá que tener un nombre, sus argumentos, un tipo de retorno, y el cuerpo; omitiendo por ejemplo la escritura concreta de llaves, e incluso el orden en que aparecen las componentes puede resultar irrelevante.
Es importante aclarar que los no terminales que no poseen producciones, son definidos de manera abstracta de forma análoga a los asumidos en la sintaxis concreta.
\begin{gather*}
\NT{abstractFun} ::= \texttt{CFun} \; \NT{id} \; ( \NT{id}, \NT{type} )_1 \ldots ( \NT{id}, \NT{type} )_n \; \NT{type} \; \NT{body}
\end{gather*}

Finalmente declaradas ambas sintaxis podemos definir una función de traducción de la concreta a la abstracta, comúnmente denominada \textbf{parser}.
Al aplicar la traducción a la función \texttt{factorial} que escribimos anteriormente, obtenemos la siguiente versión abstracta.
\begin{gather*}
\texttt{CFun} \; \texttt{factorial} \; ( \texttt{n}, \texttt{int} ) \; \texttt{int} \; body_{abs}
\end{gather*}
% IDEA: Mediante un \textit{árbol de sintaxis abstracta} se puede derivar cualquier frase del lenguaje, donde cada nodo denota uno de los elementos que la constituyen.
% IDEA: Los programas de un lenguaje formal no son simplemente cadenas de palabras, sino más bien entidades abstractas representadas por cadenas de palabras.
% IDEA: La \textit{sintaxis abstracta} de un lenguaje caracteriza el conjunto de frases abstractas que pueden ser especificadas en el, permitiendo ignorar la información irrelevante sobre como representar las frases de manera concreta, y solo capturando su naturaleza.
% IDEA: Se enfoca la atención en la información sobre la naturaleza de las frases, omitiendo detalles acerca de su representación específica como los caracteres utilizados para denotar variables y constantes, o la precedencia y asociatividad de los operadores.
% IDEA: Una vez definida la sintaxis, es posible dar semántica al lenguaje de acuerdo a las frases construidas por ella.

Claramente no cualquier cadena de caracteres conforma un programa \textit{bien escrito}, esto ocurre cuando al aplicar la función de traducción esta nos construye una frase abstracta.
Sin embargo la construcción exitosa de la frase no significa que no existan errores; solo significa que el programa está correctamente escrito según las cadenas que caracteriza la sintaxis concreta.
Todavía pueden existir errores como el uso de variables no declaradas, llamadas de funciones con una cantidad incorrecta de argumentos, o para el caso de lenguajes tipados, errores de tipo.
Para capturar esta clase de errores definimos los \textbf{chequeos estáticos} que serán propiedades que le pediremos a un programa bien escrito y que podemos verificar sin ejecutarlo.
Estos se definen sobre las frases abstractas aprovechando, como ya mencionamos, que solo cargan la información relevante de un programa concreto.
Definiendo \textit{juicios de tipado} (enunciados que establecen cuando una frase abstracta tiene un determinado tipo) podemos dar una prueba que establezca cuando un programa no tiene errores de tipo; al poder asignarle uno. De igual manera es posible definir otras clases de enunciados para el resto de los chequeos.
% IDEA: Los chequeos emplean un conjunto de \textit{reglas de tipado} para garantizar determinadas propiedades de manera estática, las cuales son independientes de la noción de evaluación del lenguaje.
% IDEA: Los cuales utilizan el \textit{árbol de sintaxis abstracta} obtenido luego de aplicar la traducción sobre la cadena de caracteres.
% IDEA: Los chequeos estáticos son propiedades que le pediremos a un programa bien escrito y que podemos verificar sin ejecutarlo.

Si chequeamos estáticamente la frase abstracta correspondiente con el ejemplo de la función \texttt{factorial}, deberíamos poder verificar que esta toma un entero y devuelve un entero, es decir que tiene tipo \textit{int en int}.
Es fundamental notar que al efectuar los chequeos sin ejecutar el programa, es posible que sigan existiendo errores que solo podrán ser capturados durante su ejecución, como la división por cero, o accesos de memoria no válidos, entre otros.

\section{Organización de la Tesis}

El escrito se divide en cinco capítulos.
En el capítulo 2 se presenta el lenguaje, donde se ilustran sus características más relevantes a través de ejemplos.
En el capítulo 3 se describe la sintaxis abstracta y se presenta la sintaxis concreta, además se detallan aspectos sobre la implementación del parser del lenguaje.
En el capítulo 4 se definen formalmente los chequeos estáticos, y se lo acompaña con derivaciones de ejemplos.
En el capítulo 5 se concluye el trabajo con un breve resumen de las actividades desarrolladas y un listado de trabajos futuros.
% IDEA: Servirá como una especie de tutorial para el lector con experiencia en programación imperativa.
% IDEA: Además se detallan las características del parser de nuestro intérprete.
% IDEA: A través de una serie de reglas de tipado, junto con derivaciones de ejemplos, se fundamentan las propiedades a verificar en un programa.
% IDEA: Se acompaña con un listado de trabajos futuros con la finalidad de encaminar la evolución a posterior del lenguaje.