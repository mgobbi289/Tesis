% Tutorial
\Lenguaje{} es un lenguaje de programación imperativo estructurado fuertemente tipado, inspirado en \Pascal{}, sobre el cual se han agregado características avanzadas como polimorfismo paramétrico, y una versión básica de polimorfismo \textit{ad hoc}.
En este capítulo, nos concentraremos en exponer sus aspectos principales a través de ejemplos, de manera que resulte como un breve tutorial de uso para el lector familiarizado con la programación imperativa.
Al encontrarse aún en etapa de desarrollo, ciertos aspectos del lenguaje se encuentran sujetos a posibles modificaciones en futuras iteraciones.
% Experiencia en Programación Imperativa
% Lenguaje Imperativo
% Influencia Pascal
% IDEA: Resulta ser el producto de la formalización del pseudocódigo empleado en la materia \Materia{}, por lo que su diseño se orienta a la enseñanza de conceptos elementales de forma clara y natural.

\section{Tipos Nativos}

El lenguaje posee tipado fuerte, por lo que toda expresión tiene asociado un tipo de datos particular.
De manera nativa, se encuentran definidos algunos tipos los cuales serán divididos en básicos y estructurados.
% IDEA: Un tipo provee un conjunto de valores los cuales pueden ser obtenidos al evaluar una expresión determinada.

\subsection{Tipos Básicos}

Un tipo básico representa un conjunto de valores individuales.
En el lenguaje se encuentran definidos los tipos numéricos, el tipo booleano, y el tipo de los caracteres.
Para cada uno existe un conjunto de constantes que representa a sus valores, y algunos operadores que permiten manipular expresiones.
\begin{itemize}
\item
Para representar los valores numéricos, el lenguaje define los tipos \lstinline[style = lang]{int} y \lstinline[style = lang]{real}. 
Los operadores aritméticos clásicos se encuentran sobrecargados para operar con valores de ambos tipos de datos.
\begin{lstlisting}[ style = lang ]
var i : int
var r : real
i := 10 / 2
r := 5.0 + 4.5
\end{lstlisting}
Una expresión de tipo entero puede ser utilizada en contextos donde se espera una expresión de tipo real.
De esta manera es posible tener expresiones donde se emplean valores numéricos de ambos tipos.
En el siguiente ejemplo se divide una sumatoria de números enteros, por un número entero, para asignar un valor a la variable real \lstinline[style = lang]{average}.
\begin{lstlisting}[ style = lang ]
var i1, i2, i3 : int
var average : real
...
average := i1 + i2 + i3 / 3
\end{lstlisting}
\item
Los valores de verdad están representados mediante el tipo \lstinline[style = lang]{bool}.
Se definen las constantes \lstinline[style = lang]{true} y \lstinline[style = lang]{false}, junto con los operadores lógicos clásicos \lstinline[style = lang]{&&}, \lstinline[style = lang]{||}, y \lstinline[style = lang]{!}.
Además se encuentran definidos los operadores de comparación tradicionales.
\begin{lstlisting}[ style = lang ]
var i : int
var b : bool
...
b := b && i < 10
\end{lstlisting}
\item
Los caracteres son representados por el tipo \lstinline[style = lang]{char}.
Se encuentran definidas cada una de las constantes \texttt{ASCII}.
A diferencia de los tipos anteriores no tenemos definidas operaciones para este tipo de datos.
\begin{lstlisting}[ style = lang ]
var c : char
c := 'z'
\end{lstlisting}
\end{itemize}

En los ejemplos previos, se ilustra la declaración y asignación de variables.
Para la declaración utilizamos la palabra clave \lstinline[style = lang]{var} seguida del identificador de la variable junto con su respectivo tipo.
Para la asignación empleamos el operador \lstinline[style = lang]{:=} para asignar el valor de una expresión particular a una variable determinada.

\subsection{Tipos Estructurados}

Un tipo estructurado permite representar colecciones de otros tipos de datos.
De manera similar a los tipos básicos, se definen operaciones específicas para acceder a los elementos que conforman al tipo.
En el lenguaje solo tenemos definidos de forma nativa a los arreglos.
% IDEA: Se caracterizan por el tipo de sus componentes, y por el método de estructuración que utilizan.

Los arreglos, representados por el tipo \lstinline[style = lang]{array}, permiten agrupar una cantidad fija de elementos de algún tipo de datos.
Cada elemento se ubica en una posición determinada, designada por índices enteros.
Para definir un arreglo es necesario detallar el tipo de sus componentes y los tamaños para cada una de sus dimensiones, los cuales deberán ser mayores a cero.

\begin{lstlisting}[ style = lang ]
var a : array [3] of real
a[0] := 10.75
a[1] := 5.5
a[2] := 20.25
\end{lstlisting}

En el ejemplo, se declara un arreglo unidimensional de reales \lstinline[style = lang]{a} con solo tres elementos.
Para acceder a una posición determinada, se utilizan corchetes \lstinline[style = lang]{[]} junto con el índice correspondiente.
Los índices válidos comienzan desde el cero, y continúan hasta el tamaño de la respectiva dimensión sin incluirlo.

Es posible crear arreglos con múltiples dimensiones especificando el tamaño de cada una en el momento de su declaración.
En el siguiente ejemplo se define un arreglo bidimensional \lstinline[style = lang]{a} con ambas dimensiones de tamaño cinco.
Para acceder a una posición determinada, simplemente se separan con comas \lstinline[style = lang]{,} las correspondientes coordenadas.

\begin{lstlisting}[ style = lang ]
var a : array [5, 5] of real
for i := 0 to 4 do
  for j := 0 to 4 do
    a[i, j] := 0.0
  od
od
\end{lstlisting}

Notar que se inicializan todas las posiciones del arreglo, utilizando la sentencia \lstinline[style = lang]{for}.
La declaración de las variables \lstinline[style = lang]{i} y \lstinline[style = lang]{j}, es implícita; donde su tipo es inferido de acuerdo a las expresiones especificadas como rangos de la iteración, y su alcance comprende el cuerpo de la sentencia.

En ciertas situaciones, se permiten emplear identificadores para representar el tamaño de algunas dimensiones de un arreglo.
De esta manera es posible abstraernos de sus tamaños concretos en el código, donde es importante no confundir esta característica que nos permite abstraer el \textit{tamaño fijo} de un arreglo, con la posibilidad de definir arreglos con dimensiones de \textit{tamaño dinámico}; lo cual no se permite en el lenguaje.
En el ejemplo, se utiliza el tamaño \lstinline[style = lang]{n} como una constante, cuyo valor será resuelto durante la ejecución del programa.

\begin{lstlisting}[ style = lang ]
var a : array [n] of real
for i := 0 to n - 1 do
  a[i] := 0.0
od
\end{lstlisting}

% Variables / Asignación
% Básicos: Numéricos / Booleanos / Caracteres
% Subtipado (Operadores Sobrecargados)
% Estructurados: Arreglos (Múltiples Dimensiones, Operadores)
% Sentencias: FOR TO

% Expresiones
% Operadores (Relevantes)

% Short - Circuit
% Estructurados: Punteros

\section{Definición de Tipos}

En el lenguaje podemos extender el conjunto de tipos de datos, definiendo nuevos tipos.
Para ello declaramos mediante la palabra clave \lstinline[style = lang]{type} seguida de un nombre el nuevo tipo de datos, que podrá construirse de tres formas distintas.
% IDEA: Se pueden declarar tipos enumerados, sinónimos de tipo, y estructuras de tipo tupla en el lenguaje.

\subsection{Enumerados}

Un tipo enumerado representa un conjunto finito de valores.
Cada valor está definido mediante un identificador único.
Para declarar un tipo enumerado se emplean las palabras claves \lstinline[style = lang]{enumerate} y \lstinline[style = lang]{end enumerate}.

\begin{lstlisting}[ style = lang ]
type day = enumerate
           Sunday
           Monday
           Tuesday
           Wednesday
           Thursday
           Friday
           Saturday
           end enumerate
\end{lstlisting}

Declarado un tipo enumerado, se permiten emplear sus valores como constantes definidas.
En el siguiente ejemplo, se inicializa la variable \lstinline[style = lang]{d} con la constante \lstinline[style = lang]{Sunday}.

\begin{lstlisting}[ style = lang ]
var d : day
d := Sunday
\end{lstlisting}

\subsection{Sinónimos}

Un sinónimo de tipo es un \textit{renombre} de un tipo existente.
En su declaración solo se requiere detallar el tipo asociado, lo cual permite utilizar este nuevo nombre para el mismo.
% IDEA: Valores de diferentes sinónimos del mismo tipo son enteramente compatibles, en el sentido que en situaciones donde se espera una expresión de cierto tipo determinado, se permite especificar una expresión de algún tipo compatible.
% IDEA: En principio un sinónimo definido es simplemente un \textit{renombre}, en el sentido que puede ser intercambiado de forma indiferente por el tipo relacionado, y el programa seguiría estando bien tipado.
% IDEA: En el momento de introducir el polimorfismo paramétrico que admite el lenguaje, se detallará la manera de declarar sinónimos paramétricos.

\begin{lstlisting}[ style = lang ]
type matrixReal = array [5, 5] of real
\end{lstlisting}

Una expresión de cierto tipo de datos puede ser empleada en contextos donde se espera un valor de uno de los sinónimos de su tipo.
En el siguiente ejemplo se declara una variable \lstinline[style = lang]{mR} del tipo \lstinline[style = lang]{matrixReal}, y se opera de manera transparente como si fuese un arreglo tradicional.
% IDEA: Un valor de un tipo es 'interpretado' como un valor del otro tipo.

\begin{lstlisting}[ style = lang ]
var mR : matrixReal
for i := 0 to 4 do
  for j := 0 to 4 do
    mR[i, j] := 0.0
  od
od
\end{lstlisting}

\subsection{Tuplas}

Una tupla es una colección finita de campos, posiblemente de diferentes tipos de datos, donde cada uno tiene asociado un identificador.
Las tuplas son utilizadas para agrupar un conjunto de valores que se relacionan de alguna manera.
Para su declaración se emplean las palabras claves \lstinline[style = lang]{tuple} y \lstinline[style = lang]{end tuple}.
Con la operación \lstinline[style = lang]{.} se accede al campo de una tupla, de acuerdo a un identificador determinado.
% IDEA: En el momento de introducir el polimorfismo paramétrico que admite el lenguaje, se detallará la manera de declarar tuplas paramétricas.

\begin{lstlisting}[ style = lang ]
type person = tuple
              initial : char,
              age     : int,
              weight  : real
              end tuple
\end{lstlisting}

En el ejemplo anterior se definió el tipo \lstinline[style = lang]{person} mediante una tupla con los campos \lstinline[style = lang]{initial}, \lstinline[style = lang]{age}, y \lstinline[style = lang]{weight}, donde sus respectivos tipos son \lstinline[style = lang]{char}, \lstinline[style = lang]{int}, y \lstinline[style = lang]{real}.
En el ejemplo siguiente se declara una variable \lstinline[style = lang]{p} del tipo introducido, y se inicializan todos los campos de manera adecuada.

\begin{lstlisting}[ style = lang ]
var p : person
p.initial := 'F'
p.age     := 30
p.weight  := 70.5
\end{lstlisting}

% Enumerados
% Sinónimos
% Tuplas (Operadores)

% Tipos con Parámetros (NO)
% Limitación para Recursión (Solo Punteros)
% Orden de Declaración

\section{Funciones y Procedimientos}

La única manera de escribir programas en nuestro lenguaje es mediante la declaración de funciones y procedimientos.
En esencia estas construcciones comprenden un bloque de código estructurado, implementado para realizar una tarea particular.
% IDEA: Se emplean para favorecer la modularidad del programa, y promover la reutilización de código.

Una función realiza una computación de acuerdo a un conjunto de parámetros, y al finalizar retorna siempre un resultado en el contexto donde fue llamada.
Su comportamiento es determinado solamente por los valores de los argumentos que recibe, donde tampoco podrá modificar el estado de los mismos.
% Una función no modifica el estado de sus argumentos.
% IDEA: Son independientes del estado del programa.

\begin{lstlisting}[ style = lang ]
{@ PRE: n >= 0 @}
fun factorial ( n : int ) ret fact : int
  fact := 1
  for i := 2 to n do
    fact := fact * i
  od
end fun
\end{lstlisting}

En el ejemplo se muestra la implementación de la función \lstinline[style = lang]{factorial}, que calcula el factorial de un número entero positivo \lstinline[style = lang]{n}.
La variable de retorno \lstinline[style = lang]{fact} almacena la productoria de números, y al finalizar la ejecución de la función, se retorna su valor al contexto donde se efectuó la llamada.

Un procedimiento realiza una computación de acuerdo a un conjunto de parámetros de lectura, para modificar un conjunto de parámetros de escritura.
Su comportamiento es determinado solamente por los parámetros que recibe donde cada uno lleva un decorado que indica si es de lectura \lstinline[style = lang]{in}, de escritura \lstinline[style = lang]{out}, o ambas \lstinline[style = lang]{in/out}.
Un procedimiento no modifica el estado de los parámetros de lectura, y tampoco consulta el estado de los parámetros de escritura.

\begin{lstlisting}[ style = lang ]
proc initialize ( in e : int, out a : array [10] of int )
  for i := 9 downto 0 do
    a[i] := e
  od
end proc
\end{lstlisting}

En el ejemplo se implementa el procedimiento \lstinline[style = lang]{initialize}, que inicializa un arreglo de enteros de acuerdo a un valor determinado.
Notar que el parámetro de lectura \lstinline[style = lang]{e} solo ocurre del lado derecho de la asignación, mientras que el parámetro de escritura \lstinline[style = lang]{a} solo ocurre del lado izquierdo.
% IDEA: Una vez finalizada la ejecución del procedimiento, el arreglo de escritura \lstinline[style = lang]{a} se encontrará inicializado de acuerdo al valor de lectura \lstinline[style = lang]{e}.

Previamente cuando se presentó el tipo de los arreglos, se mencionó que el lenguaje permite la utilización de identificadores para describir que un arreglo tiene un tamaño fijo que aun no se ha determinado (lo cual no debe confundirse con arreglos de tamaños dinámicos que el lenguaje no posee).
La introducción de tales identificadores solo es posible durante la declaración del prototipo de una función o un procedimiento, y siempre en el contexto de un parámetro de tipo \lstinline[style = lang]{array}.
Adicionalmente este identificador podrá ser utilizado como una constante en el respectivo cuerpo de la construcción declarada.

\begin{lstlisting}[ style = lang ]
{@ PRE: 0 <= i, j < n @}
proc swap ( in/out a : array [n] of int, in i, j : int )
  var temp : int
  temp := a[i]
  a[i] := a[j]
  a[j] := temp
end proc
\end{lstlisting}

En el ejemplo se declara el procedimiento \lstinline[style = lang]{swap}, que intercambia los valores de las posiciones \lstinline[style = lang]{i} y \lstinline[style = lang]{j} del arreglo \lstinline[style = lang]{a}.
Notar que el parámetro \lstinline[style = lang]{a} de tipo \lstinline[style = lang]{array} introduce el identificador \lstinline[style = lang]{n}, lo cual nos permite abstraer el tamaño fijo del arreglo.
En consecuencia, se generaliza la declaración del procedimiento ya que podrá ser aplicado a arreglos de una dimensión con cualquier tamaño; a diferencia de si en lugar de \lstinline[style = lang]{n} se hubiese precisado algún tamaño puntual utilizando un entero.
Es importante notar que aunque el procedimiento puede ser aplicado a arreglos de cualquier tamaño, estos serán fijos.
Lo cual significa que el tamaño del arreglo \lstinline[style = lang]{a} no debe ser pensado como dinámico, sino como un tamaño fijo \lstinline[style = lang]{n} aun no determinado; lo cual ocurrirá al momento de la aplicación del procedimiento.

El lenguaje soporta recursión.
La declaración de funciones y procedimientos permite definiciones recursivas, donde en el respectivo cuerpo es posible llamar a la construcción que está siendo definida.
En el siguiente ejemplo se muestra una implementación alternativa para la función \lstinline[style = lang]{factorial} utilizando recursión.
% IDEA: Notar que para la llamada a una función solo se debe especificar su nombre, junto con las expresiones asociadas a cada uno de sus argumentos.

\begin{lstlisting}[ style = lang ]
{@ PRE: n >= 0 @}
fun factorial ( n : int ) ret fact : int
  if n >= 2 then
    fact := n * factorial(n - 1)
  else
    fact := 1
  fi
end fun
\end{lstlisting}

El orden de declaración de funciones y procedimientos es fundamental, ya que solo se permite efectuar una llamada a funciones o procedimientos que hayan sido definidos con anterioridad.
En particular, no es posible definir funciones o procedimientos mediante recursión mutua.
% IDEA: Los identificadores empleados para nombrar a funciones y procedimientos deben ser únicos, para reconocer a que construcción se intenta invocar durante una llamada.

% Orden de Declaración
% Recursión Función
% Llamada de Función
% Sentencias (IF)
% Argumentos / Retorno (Pasaje por Valor)
% IN / OUT / INOUT

% Recursión Procedimiento
% Sentencias (FOR TO, FOR DOWNTO)

% Llamada de Procedimiento
% Polimorfismo (NO)
% Sentencias (WHILE)

\section{Polimorfismo Paramétrico}

El polimorfismo es la característica de los lenguajes de programación mediante la cual se pueden definir funciones y procedimientos de manera genérica para diferentes tipos de datos.
Consideremos los ejemplos previos de inicializar un arreglo con un elemento dado, y de intercambiar posiciones de elementos en un arreglo.
En las definiciones anteriores el arreglo es de enteros.
Sin embargo las tareas de inicializar o intercambiar elementos también pueden ser útiles en arreglos de otros tipos de datos.
Podríamos entonces redefinir ambos procedimientos con los nombres \lstinline[style = lang]{initializeInt}, \lstinline[style = lang]{swapInt}, y luego definir procedimientos \textit{idénticos} a estos, pero ahora recibiendo arreglos de booleanos, llamándolos por ejemplo \lstinline[style = lang]{initializeBool}, \lstinline[style = lang]{swapBool}.

\begin{lstlisting}[ style = lang ]
proc initializeInt ( in e : int, out a : array [n] of int )
  for i := n - 1 downto 0 do
    a[i] := e
  od
end proc
\end{lstlisting}

\begin{lstlisting}[ style = lang ]
proc initializeBool ( in e : bool, out a : array [n] of bool )
  for i := n - 1 downto 0 do
    a[i] := e
  od
end proc
\end{lstlisting}

\begin{lstlisting}[ style = lang ]
{@ PRE: 0 <= i, j < n @}
proc swapInt ( in/out a : array [n] of int, in i, j : int )
  var temp : int
  temp := a[i]
  a[i] := a[j]
  a[j] := temp
end proc
\end{lstlisting}

\begin{lstlisting}[ style = lang ]
{@ PRE: 0 <= i, j < n @}
proc swapBool ( in/out a : array [n] of bool, in i, j : int )
  var temp : bool
  temp := a[i]
  a[i] := a[j]
  a[j] := temp
end proc
\end{lstlisting}

Si luego quisiéramos implementar la misma tarea pero para otro tipo de datos, deberíamos volver a escribir una vez más el mismo código.
El \textit{polimorfismo paramétrico} permite definir una única vez estos procedimientos de manera genérica, permitiendo que sean utilizados para cualquier tipo de datos.

\begin{lstlisting}[ style = lang ]
proc initialize ( in e : T, out a : array [n] of T )
  for i := n - 1 downto 0 do
    a[i] := e
  od
end proc
\end{lstlisting}

\begin{lstlisting}[ style = lang ]
{@ PRE: 0 <= i, j < n @}
proc swap ( in/out a : array [n] of T, in i, j : int )
  var temp : T
  temp := a[i]
  a[i] := a[j]
  a[j] := temp
end proc
\end{lstlisting}

Notemos que las tareas de inicializar e intercambiar son independientes del tipo de datos almacenado por el arreglo.
La tarea de inicializar recorre cada posición del arreglo y almacena el valor dado, y la tarea de intercambiar almacena los correspondientes valores en las posiciones intercambiadas; sea cual sea el tipo de datos de los elementos del arreglo ambas tareas son idénticas.
La característica que permite definir una \textit{misma} función o procedimiento para cualquier tipo de datos, se denomina \textbf{polimorfismo paramétrico}.
% IDEA: Realizar llamada a procedimientos con polimorfismo paramétrico

La idea de la declaración de funciones y procedimientos cuyos comportamientos son independientes del tipo de datos a los que se aplican, puede ser adaptada para la declaración de tipos.
Consideremos el siguiente ejemplo donde se define el tipo tupla \lstinline[style = lang]{pair}.
La declaración de un tipo paramétrico nos permite generalizar su estructura, la cual será independiente de los tipos que serán instanciados en sus parámetros.

\begin{lstlisting}[ style = lang ]
type pair of (A, B) = tuple
                      first  : A,
                      second : B
                      end tuple
\end{lstlisting}

El polimorfismo paramétrico permite definir una única vez el tipo, que luego podrá ser instanciado de manera apropiada según sea necesario.
Si estuviésemos trabajando con puntos en un plano, por ejemplo, una forma para representarlos sería mediante el tipo \lstinline[style = lang]{pair of (real, real)} donde cada campo corresponde a una coordenada del punto respectiva a los ejes del plano.

\begin{lstlisting}[ style = lang ]
fun first (p : pair of (A, B)) ret fst : A
  fst := p.first
end fun
\end{lstlisting}

\begin{lstlisting}[ style = lang ]
fun second (p : pair of (A, B)) ret snd : B
  snd := p.second
end fun
\end{lstlisting}

Combinando los conceptos presentados sobre el polimorfismo paramétrico, podemos implementar los ejemplos anteriores \lstinline[style = lang]{first} y \lstinline[style = lang]{second}.
La operación de obtener cualquiera de los campos de la tupla, es independiente del tipo de datos almacenados en la misma.
Notar que ambos conceptos se complementan de cierta manera, donde funciones y procedimientos polimórficos pueden hacer uso de los tipos paramétricos.

% TAMAÑO DINÁMICO : Inicializar / Swap
% TAMAÑOS : en los arreglos podemos introducir una variable en el lugar donde se indica el tamaño, explicamos lo que permite hacer eso, y luego podemos mencionar que lo podemos pensar como un caso de type family

% Procedimientos Polimórficos ( Variables de Tipo / Instancia )
% Tipos con Parámetros (Tuplas)
% Funciones Polimórficas ( Variables de Tipo / Instancia )

% Procedimientos Polimórficos ( Tamaños Dinámicos )
% Limitaciones: Forzar Variables, Tamaños Concretos
% Funciones Polimórficas ( Tamaños Dinámicos )

\section{Polimorfismo \textit{Ad Hoc}}

En muchas ocasiones podemos identificar que una función o un procedimiento podrían ser implementados de manera genérica pero no para cualquier tipo de datos, sino para ciertos tipos que comparten alguna característica, en consecuencia no es posible utilizar \textit{polimorfismo paramétrico}.
Consideremos las siguientes implementaciones de \lstinline[style = lang]{belongs} y \lstinline[style = lang]{selectionSort}.
La primera decide si un valor determinado pertenece a un arreglo y la segunda permite ordenar un arreglo de menor a mayor.

\begin{lstlisting}[ style = lang ]
fun belongs ( e : int, a : array [n] of int ) ret b : bool
  var i : int
  i := 0
  b := false
  while !b && i < n do
    b := a[i] == e
    i := i + 1
  od
end fun
\end{lstlisting}

\begin{lstlisting}[ style = lang ]
proc selectionSort ( in/out a : array [n] of int )
  var minPos : int
  for i := 0 to n - 1 do
    minPos := i
    for j := i + 1 to n - 1 do
      if a[j] < a[minPos] then minPos := j fi
    od
    swap(a, i, minPos)
  od
end proc
\end{lstlisting}

En ambas implementaciones los elementos son de tipo entero.
Sin embargo podríamos definir las mismas operaciones para otros tipos de datos, como por ejemplo, caracteres.
Se tendrían que redefinir las anteriores con los nombres \lstinline[style = lang]{belongsInt} y \lstinline[style = lang]{selectionSortInt}, y declarar de manera idéntica las operaciones \lstinline[style = lang]{belongsChar} y \lstinline[style = lang]{selectionSortChar} donde solo cambiaríamos \lstinline[style = lang]{int} por \lstinline[style = lang]{char} en los tipos de los parámetros.
Con un trabajo tediosamente repetitivo se podrían dar declaraciones para todos los tipos que tengan definidas las operaciones de comparación; aunque no sería posible para aquellos que no las tengan definidas.

El \textit{polimorfismo ad hoc} nos permite escribir de manera genérica una función o un procedimiento donde la tarea que realiza sólo está bien definida para algunos tipos.
Además esta tarea puede no ser la misma dependiendo del tipo.

En el lenguaje se definen una serie de clases las cuales pueden ser pensadas como una especie de interfaz que caracteriza algún comportamiento.
Un tipo es una instancia de una clase, cuando implementa el comportamiento que la clase describe.
El lenguaje sólo incorpora de forma nativa las clases \lstinline[style = lang]{Eq} y \lstinline[style = lang]{Ord}, y no existe posibilidad de declarar nuevas clases.
La primera representa a los tipos que tienen alguna noción de igualdad, y sus operaciones definidas comprenden al \lstinline[style = lang]{==} y \lstinline[style = lang]{!=}.
La segunda representa a los tipos que poseen alguna relación de orden, y sus operaciones definidas comprenden al \lstinline[style = lang]{<}, \lstinline[style = lang]{<=}, \lstinline[style = lang]{>=}, \lstinline[style = lang]{>}.

\begin{lstlisting}[ style = lang ]
fun belongs ( e : T, a : array [n] of T ) ret b : bool
where (T : Eq)
  var i : int
  i := 0
  b := false
  while !b && i < n do
    b := a[i] == e
    i := i + 1
  od
end fun
\end{lstlisting}

\begin{lstlisting}[ style = lang ]
proc selectionSort ( in/out a : array [n] of T )
where (T : Ord)
  var minPos : int
  for i := 0 to n - 1 do
    minPos := i
    for j := i + 1 to n - 1 do
      if a[j] < a[minPos] then minPos := j fi
    od
    swap(a, i, minPos)
  od
end proc
\end{lstlisting}

En los ejemplos, notemos que las tareas de comprobar pertenencia y ordenar elementos solo dependen de las operaciones de comparación soportadas por el tipo de datos almacenado por el arreglo.
La función que comprueba la pertenencia recorre cada posición del arreglo, y aplica la operación \lstinline[style = lang]{==} para verificar si los elementos coinciden, por lo tanto el tipo de los elementos del arreglo deberá tener instancia de \lstinline[style = lang]{Eq}.
El procedimiento que ordena los elementos recorre el arreglo intercambiando las posiciones de sus valores, de acuerdo al resultado de la operación \lstinline[style = lang]{<}, es decir el tipo de los elementos deberá tener instancia de \lstinline[style = lang]{Ord}.

% En un futuro desarrollo del lenguaje, será posible declarar instancias para los tipos definidos por el usuario.

% Clases / TypeClasses
% Operadores de Orden e Igualdad
% Restricciones de Clase
% Llamada con Restricciones
% Instancias

\section{Memoria Dinámica}

El lenguaje permite manipular explícitamente la \textit{memoria dinámica} mediante un tipo de datos especial, que llamaremos puntero.
Supongamos que deseamos definir el tipo correspondiente a las listas.
Una lista permite representar una colección de elementos de algún tipo de datos, cuyo tamaño es variable; lo cual significa que su tamaño crece tanto como sea necesario, de acuerdo a la cantidad de elementos almacenados.
Todos los tipos presentados hasta el momento utilizan una cantidad fija de memoria, la cual no puede ser modificada en tiempo de ejecución.
Recordemos una vez más que los arreglos implementados en el lenguaje tienen un tamaño fijo al momento de la ejecución.
En este aspecto el uso de punteros resulta fundamental, ya que permiten reservar y liberar memoria en la medida que sea necesario durante la ejecución del programa.

\begin{lstlisting}[ style = lang ]
type node of (T) = tuple
                   elem : T,
                   next : pointer of node of (T)
                   end tuple

type list of (T) = pointer of node of (T)
\end{lstlisting}

Un puntero, representado por el tipo \lstinline[style = lang]{pointer}, indica el lugar en memoria donde se aloja un elemento de cierto tipo.
Combinando la declaración de tipos utilizando tuplas con punteros, se permiten declarar definiciones recursivas en el lenguaje; lo cual resulta indispensable para la declaración de las listas.
En el ejemplo anterior se declara una \textit{lista enlazada} denominada \lstinline[style = lang]{list}, la cual se compone de una sucesión de nodos \lstinline[style = lang]{node} que se integran por los campos \lstinline[style = lang]{elem} de cierto tipo paramétrico \lstinline[style = lang]{T}, y \lstinline[style = lang]{next} el cual referencia al siguiente nodo en la lista.

\begin{lstlisting}[ style = lang ]
var p : pointer of int
alloc(p)
...
#p := #p + 1
...
free(p)
\end{lstlisting}

En el lenguaje se definen tres operaciones para manipular punteros.
El procedimiento nativo \lstinline[style = lang]{alloc} toma una variable de tipo puntero, y le asigna la dirección de un nuevo bloque de memoria, cuyo tamaño estará determinado por el tipo de la variable.
El operador \lstinline[style = lang]{#} permite acceder al bloque de memoria apuntado por el puntero.
El procedimiento nativo \lstinline[style = lang]{free} toma una variable de tipo puntero, y libera el respectivo bloque de memoria referenciado.

Retomando el ejemplo de la \textit{lista enlazada}, los procedimientos \lstinline[style = lang]{empty} y \lstinline[style = lang]{addL} son utilizados para construir valores del tipo en cuestión.
% IDEA: Garbage Collector
% IDEA: Aliasing

\begin{lstlisting}[ style = lang ]
proc empty ( out l : list of (T) )
  l := null
end proc
\end{lstlisting}

El procedimiento \lstinline[style = lang]{empty} construye una lista vacía.
La constante \lstinline[style = lang]{null} representa un puntero que no referencia un lugar de memoria válido.
En el ejemplo, la constante representa una lista que no posee ningún nodo.
% IDEA: Es utilizada para indicar que un puntero no debe ser accedido, ya que no referencia ninguna posición de memoria válida.

\begin{lstlisting}[ style = lang ]
proc addL ( in e : T, in/out l : list of (T) )
  var p : pointer of node of (T)
  alloc(p)
  p->elem := e
  p->next := l
  l := p
end proc
\end{lstlisting}

El procedimiento \lstinline[style = lang]{addL} agrega un elemento al principio de la lista.
Se reserva memoria para un nodo, se inicializan sus campos, y se modifica la lista para que señale al elemento en cuestión.
El operador \lstinline[style = lang]{->} es una notación conveniente para acceder al campo de una tupla señalada por un puntero; lo cual evita tener que emplear ambos operadores de acceso a punteros y tuplas, en lugar de escribir \lstinline[style = lang]{#p.elem} simplemente se denota \lstinline[style = lang]{p->elem} para acceder a un elemento de la lista.

% Punteros (Operadores * ->, Procedimientos alloc free)
% Reservar / Liberar Memoria
% Garbage Collector (NO)
% NULL
% Recursión con Punteros
% Aliasing