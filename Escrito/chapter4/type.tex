% Chequeos para Tipos
\subsection{Chequeos para Tipos}

Las primeras reglas que definiremos serán sobre los tipos que ocurren en el programa, estas buscan verificar propiedades tales como el uso adecuado de los tipos definidos por el usuario, y limitar la ocurrencia de variables de tipo, e identificadores de tamaños dinámicos de arreglos. %, en entornos donde su especificación sea válida.
%Debido a esto, a continuación describiremos 
Comenzamos definiendo los distintos contextos que serán necesarios para la definición de los juicios y las reglas.

\subsubsection{Contextos para Declaración de Tipos}

Una definición de tipo \textit{typedecl} consiste en alguna de las siguientes tres construcciones sintácticas; un tipo enumerado, un sinónimo de tipo, o una estructura de tipo tupla.
Cuando una de estas declaraciones se encuentre \textit{bien tipada} su información será almacenada en el contexto adecuado.
Habrá un contexto diferente para cada una de las categorías de tipos que se pueden definir en el lenguaje.
Además, estos conjuntos tendrán una serie de invariantes que definirán cuando un contexto está correctamente construido. %deberán ser respetadas a lo largo del análisis de un programa.

\begin{itemize}
    \item $\T{enum} \; tn = cn_1, cn_2, \ldots, cn_m$
    \item $\T{syn} \; tn \; \T{of} \; tv_1, \ldots, tv_l = \theta$
    \item $\T{tuple} \; tn \; \T{of} \; tv_1, \ldots, tv_l = fn_1 : \theta_1, \ldots, fn_m : \theta_m$
\end{itemize}

En el caso de la declaración de un tipo enumerado, se debe almacenar el nombre del tipo definido junto con el listado de constantes enumeradas en su cuerpo.
Una invariante que se debe respetar en este contexto, es que los nombres de constantes deben ser únicos.
Esto significa que no pueden ser repetidos dentro de una misma definición, ni tampoco ocurrir en otras.
\begin{gather*}
\pi_{e} =
\{ 
(tn, \{ cn_1, \ldots, cn_m \} ) \mid 
tn \in \NT{tname} 
\wedge 
cn_i \in \NT{cname}
\}
\end{gather*}

Para los sinónimos, además del nombre, se deben guardar las variables de tipo utilizadas como argumentos, junto con el tipo que lo define.
En este contexto, la invariante debe asegurar que dentro de una declaración no se repitan los identificadores empleados para representar a sus parámetros.
\begin{gather*}
\pi_{s} =
\{
(tn, \{ tv_1, \ldots, tv_l \} , \theta) \mid 
tn \in \NT{tname}
\wedge
tv_i \in \NT{typevariable}
\wedge
\theta \in \NT{type}
\}
\end{gather*}

Finalmente, para las tuplas, tenemos que almacenar su nombre, sus argumentos de tipo, y los distintos campos especificados en su definición.
En esta situación, además de evitar la repetición de variables de tipo, se tiene que asegurar que los identificadores de campo sean únicos dentro del cuerpo de la declaración.
\begin{gather*}
\pi_{t} =
\{
(tn, \{ tv_1, \ldots, tv_l \}, \{ fd_1, \ldots, fd_m \} ) \mid
tn \in \NT{tname} 
\wedge
tv_i \in \NT{typevariable}
\wedge
fd_j \in \NT{field}
\}
\end{gather*}

Estos tres contextos se encargarán de almacenar toda la información relacionada con los tipos declarados.% por el usuario en un programa.
A todas las condiciones de consistencia mencionadas anteriormente se le tiene que sumar una última, los nombres de tipos definidos deben ser únicos.
Es decir, que no puede haber más de una definición para el mismo identificador de tipo entre los distintos contextos.

\subsubsection{Contexto para Variables de Tipo}

%Cuando se analiza un tipo, es necesario llevar un registro de las variables de tipo introducidas en el alcance actual.
En general, la definición de cualquier regla de tipado para un tipo deberá tener un listado de las variables de tipo introducidas en el alcance actual.
Ya sea como parámetro de tipo en una declaración de tipo, o como tipo polimórfico para algún parámetro de una función o un procedimiento, solo se deben permitir el uso de estas variables cuando se han introducido previamente en el código.
%De lo contrario, durante la ejecución del programa no se podría obtener el tipo concreto que las mismas representarían.
\begin{gather*}
\pi_{tv} \subseteq \NT{typevariable}
\end{gather*}

\subsubsection{Contexto para Identificadores de Tamaños Dinámicos}

Similar como sucede con las variables de tipo, hay una necesidad de llevar registro de todos los identificadores de tamaños dinámicos que son introducidos. % a lo largo del análisis.
En el prototipo de funciones y procedimientos, se pueden especificar tamaños dinámicos para las dimensiones de arreglos.
Esto permite que dentro de sus cuerpos, se puedan utilizar estos identificadores para declarar nuevos arreglos cuyas dimensiones coincidirán con los arreglos correspondientes en el encabezado. %, independientemente del tamaño concreto de los mismos.
\begin{gather*}
\pi_{sn} \subseteq \NT{sname}
\end{gather*}

\subsubsection{Reglas para Tipos}

En un programa, hay tres situaciones diferentes donde se puede escribir un tipo, y
en base a esta, las propiedades que se deben verificar varían levemente.
Un tipo puede ocurrir en una declaración de tipo, en una definición del prototipo de una función o procedimiento, e incluso en la declaración de una variable en algún cuerpo de una función o procedimiento.
%En todos estos contextos distintos, el uso de una variable de tipo o un tamaño dinámico de arreglo, se debe permitir solo si durante una hipotética ejecución del programa se cuenta con la información suficiente para dar un tipo o valor concreto, respectivamente, a esta construcción.

Cuando nos encontramos analizando un tipo, utilizaremos la siguiente notación para denotar que el tipo $\theta$ es válido en el contexto de los tipos definidos; enumerados $\pi_{e}$, sinónimos $\pi_{s}$, y tuplas $\pi_{t}$.
Sumados a los mismos, los conjuntos $\pi_{tv}$ de variables de tipo determinará cuando la escritura de un tipo polimórfico particular está permitida, y tenemos además $\pi_{sn}$ para los identificadores de tamaños dinámicos.
Utilizando una notación más compacta, comúnmente haremos referencia al contexto $\PI{T}$ para representar a la anterior tripla de contextos.
Mientras que emplearemos $\overline{\pi}$ para simbolizar a los últimos dos conjuntos.
Esta salvedad la tendremos para facilitar la lectura de las reglas, y poder concentrarnos propiamente en las derivaciones.
\begin{align*}
\pi_{e}, \pi_{s}, \pi_{t}
\quad
&\DASH[\mathclap{\pi_{tv}, \pi_{sn}}]{t}
\quad
\theta
\\
\PI{T}
\quad
&\DASH[\overline{\pi}]{t}
\quad
\theta
\end{align*}

Para decidir si uno de estos \textit{juicios} es válido, tenemos que proveer una derivación, o prueba, utilizando las reglas que definiremos a continuación.
%Con la aplicación sucesiva de las mismas, se pueden construir pruebas que demuestran las distintas propiedades requeridas para que un tipo en un programa sea considerado estáticamente correcto.

Comenzaremos con los tipos básicos del lenguaje.
La prueba de los mismos es inmediata, ya que su regla no presenta ninguna premisa.
Por lo tanto, todo tipo básico es considerado un tipo correcto.

\begin{TRegla}
\label{TBasico}
Básicos
\begin{prooftree}
\AxiomC{}
\RightLabel
{
\quad cuando $\theta \in \{ \T{int}, \T{real}, \T{bool}, \T{char} \}$
}
\UnaryInfC
{$
\PI{T} \DASH[\overline{\pi}]{t} \theta
$}
\end{prooftree}
\end{TRegla}

Un puntero será correcto, siempre que el tipo del valor al que hace referencia sea correcto.
Notar que la premisa de la regla requiere de la prueba de un tipo estructuralmente menor al inicial.

\begin{TRegla}
\label{TPuntero}
Punteros
\begin{prooftree}
\AxiomC
{$
\PI{T} \DASH[\overline{\pi}]{t} \theta
$}
\UnaryInfC
{$
\PI{T} \DASH[\overline{\pi}]{t} \T{pointer} \; \theta
$}
\end{prooftree}
\end{TRegla}

Para los arreglos, se necesitarán verificar los tamaños de sus dimensiones, junto con el tipo de valores que almacenará.
Notar que para analizar los primeros, solo se requiere la información de los identificadores de tamaños dinámicos presentes en el contexto local. %alcance actual.

\begin{TRegla}
\label{TArreglo}
Arreglos
\begin{prooftree}
\AxiomC
{$
\DASH[\pi_{sn}]{as} as_i
$}
\AxiomC
{$
\PI{T} \DASH[\pi_{tv}, \pi_{sn}]{t} \theta
$}
\BinaryInfC
{$
\PI{T} \DASH[\pi_{tv}, \pi_{sn}]{t} \T{array} \; as_1, \ldots, as_n \; \T{of} \; \theta
$}
\end{prooftree}
\end{TRegla}

Para las dimensiones de un arreglo, se pueden especificar sus tamaños de dos formas.
Si el mismo es un valor natural, la verificación es inmediata ya que cualquier natural representa un tamaño de arreglo válido.
En cambio, cuando se utilizan identificadores de tamaños dinámicos, es necesario comprobar que haya sido previamente introducido, %que su empleo sea válido.
el único lugar de un programa donde se permiten introducir esta clase de identificadores es en el prototipo de funciones y procedimientos.
%Luego, estos elementos podrán ser utilizados en sus respectivos cuerpos para declarar nuevos arreglos.

\begin{TRegla}
\label{TConcreto}
Tamaños Concretos
\begin{prooftree}
\AxiomC{}
\RightLabel{\quad cuando $as \in \NT{natural}$}
\UnaryInfC
{$
\DASH[\pi_{sn}]{as} as
$}
\end{prooftree}
\end{TRegla}

\begin{TRegla}
\label{TDinamico}
Tamaños Dinámicos
\begin{prooftree}
\AxiomC
{$
as \in \pi_{sn}
$}
\UnaryInfC
{$
\DASH[\pi_{sn}]{as} as
$}
\end{prooftree}
\end{TRegla}

Al utilizar una variable de tipo, es necesario verificar que su uso sea válido en el alcance actual.
Similar a los identificadores de tamaños dinámicos, se permiten introducir esta clase de variables en el encabezado de funciones y procedimientos, lo que luego posibilitará emplearlas en sus respectivos cuerpos.
Adicionalmente, existe la posibilidad de declarar un tipo de forma paramétrica, lo que concede la capacidad de usar las variables de tipo, especificadas como argumento, en la definición del nuevo tipo.

\begin{TRegla}
\label{TVariable}
Variables de Tipo
\begin{prooftree}
\AxiomC
{$
tv \in \pi_{tv}
$}
\UnaryInfC
{$
\PI{T} \DASH[\pi_{tv}, \pi_{sn}]{t} tv
$}
\end{prooftree}
\end{TRegla}

Las reglas para los tipos definidos, pueden separarse en dos categorías de acuerdo si los mismos poseen argumentos de tipo, o no.
La verificación de un tipo definido no parametrizado, consiste simplemente de constatar que su nombre se encuentra declarado en alguno de los contextos de tipos correspondientes.
Evidentemente, hay que asegurar que en su definición no se haya especificado ningún argumento de tipo.

\begin{TRegla}
\label{TEnumerado}
Tipos Enumerados
\begin{prooftree}
\AxiomC
{$
(tn, \{ cn_1, \ldots, cn_m \}) \in \pi_{e}
$}
\UnaryInfC
{$
\pi_{e}, \pi_{s}, \pi_{t} \DASH[\overline{\pi}]{t} tn
$}
\end{prooftree}
\end{TRegla}

\begin{TRegla}
\label{TSinonimo}
Sinónimos sin Argumentos
\begin{prooftree}
\AxiomC
{$
(tn, \emptyset,\theta) \in \pi_{s}
$}
\UnaryInfC
{$
\pi_{e}, \pi_{s}, \pi_{t} \DASH[\overline{\pi}]{t} tn
$}
\end{prooftree}
\end{TRegla}

\begin{TRegla}
\label{TTupla}
Tuplas sin Argumentos
\begin{prooftree}
\AxiomC
{$
(tn, \emptyset, \{ fd_1, \ldots, fd_m \}) \in \pi_{t}
$}
\UnaryInfC
{$
\pi_{e}, \pi_{s}, \pi_{t} \DASH[\overline{\pi}]{t} tn
$}
\end{prooftree}
\end{TRegla}

En cambio, para un tipo parametrizado, es necesario realizar unas verificaciones adicionales.
En particular, se deben validar todos los tipos especificados como argumentos del mismo, y asegurar que la cantidad de argumentos coincida con el número de parámetros declarados en la definición del tipo.

\begin{TRegla}
\label{TSinonimoP}
Sinónimos con Argumentos
\begin{prooftree}
\AxiomC
{$
(tn, \{ tv_1, \ldots, tv_l \}, \theta) \in \pi_{s}
$}
\AxiomC
{$
\pi_{e}, \pi_{s}, \pi_{t} \DASH[\overline{\pi}]{t} \theta_i
$}
\BinaryInfC
{$
\pi_{e}, \pi_{s}, \pi_{t} \DASH[\overline{\pi}]{t} tn \; \T{of} \; \theta_1, \ldots, \theta_l
$}
\end{prooftree}
\end{TRegla}

\begin{TRegla}
\label{TTuplaP}
Tuplas con Argumentos
\begin{prooftree}
\AxiomC
{$
(tn, \{ tv_1, \ldots, tv_l \}, \{ fd_1, \ldots, fd_m \}) \in \pi_{t}
$}
\AxiomC
{$
\pi_{e}, \pi_{s}, \pi_{t} \DASH[\overline{\pi}]{t} \theta_i
$}
\BinaryInfC
{$
\pi_{e}, \pi_{s}, \pi_{t} \DASH[\overline{\pi}]{t} tn \; \T{of} \; \theta_1, \ldots, \theta_l
$}
\end{prooftree}
\end{TRegla}

\subsubsection{Ejemplo de Prueba}

%A continuación, presentamos como es la derivación para una prueba de corrección de un tipo determinado del lenguaje.
Supongamos los siguientes contextos donde no hay declarados tipos enumerados, ni sinónimos de tipo, y solo tenemos definido el tipo tupla \textit{node} parametrizado en \textit{Z}.
Notar que el campo \textit{elem} es de tipo variable \textit{Z}, y \textit{next} es un puntero al mismo tipo.
\begin{gather*}
\pi_e = \emptyset
\\
\pi_s = \emptyset
\\
\pi_t = \{ (node, \{ Z \}, \{ elem: Z, next: \T{pointer} \; node \; \T{of} \; Z \} ) \}
\end{gather*}

Sumados a los contextos anteriores, también se encuentran definidos los siguientes conjuntos.
No hay ningún tamaño dinámico declarado en el alcance actual, y solo se ha introducido la variable de tipo \textit{A}.
\begin{gather*}
\pi_{sn} = \emptyset
\\
\pi_{tv} = \{ A \}
\end{gather*}

Si nos adelantamos un poco, y quisiéramos probar la declaración del sinónimo de tipo $\T{syn} \; list \; \T{of} \; A = \T{pointer} \; node \; \T{of} \; A$, entonces una parte de la prueba consistirá en demostrar la validez del tipo que ocurre dentro de la definición.
Por lo tanto, con los contextos previos, se puede realizar la siguiente derivación.
Notar el uso de las reglas para punteros, tuplas definidas, y variables de tipo.

\begin{Prueba}
\label{PTPointerNode}
Demostración de corrección para el tipo \emph{puntero a nodo}.
\begin{prooftree}
\AxiomC
{$
(node, \{ Z \}, \{ elem: Z, next: \T{pointer} \; node \; \T{of} \; Z \}) \in \pi_{t}
$}
\AxiomC
{$
A \in \pi_{tv}
$}
\RightLabel{\RULE{\ref{TVariable}}}
\UnaryInfC
{$
\pi_{e}, \pi_{s}, \pi_{t} \DASH[\pi_{tv}, \pi_{sn}]{t} A
$}
\RightLabel{\RULE{\ref{TTuplaP}}}
\BinaryInfC
{$
\pi_{e}, \pi_{s}, \pi_{t} \DASH[\pi_{tv}, \pi_{sn}]{t} node \; \T{of} \; A
$}
\RightLabel{\RULE{\ref{TPuntero}}}
\UnaryInfC
{$
\pi_{e}, \pi_{s}, \pi_{t} \DASH[\pi_{tv}, \pi_{sn}]{t} \T{pointer} \; node \; \T{of} \; A
$}
\end{prooftree}
\end{Prueba}

\subsection{Variables de Tipo Libres}

%Una variable de tipo solo puede ser utilizada en determinados contextos.
%Es fundamental que durante la ejecución de un programa, se cuente con la suficiente cantidad de información para poder deducir el tipo concreto que esta clase de construcciones representarán.
%Debido a esto, la especificación de variables de tipo se permite bajo ciertas condiciones particulares.

En una declaración de tipo, las únicas variables que se pueden, y deben, utilizar son las que fueron previamente introducidas como parámetros de la declaración. Es importante remarcar que no existe la posibilidad de introducir una variable de tipo que luego no ocurre en la declaración de tipo. Lo cual es una diferencia con los prototipos de funciones y procedimientos en donde se permite introducir variables de tipo que luego pueden no ocurrir nunca en su respectivo cuerpo.
%En cambio, para el argumento de una función o procedimiento, se permite introducir cualquier variable de tipo para su definición.
%Luego, cuando se declara una variable en el cuerpo, existe la posibilidad de utilizar las variables de tipo introducidas en el prototipo de la función o procedimiento correspondiente.
\begin{gather*}
FTV: \NT{type} \rightarrow \{ \; \NT{typevariable} \; \}
\end{gather*}

Debido a todo esto, necesitamos definir cuando una variable de tipo es considerada \textit{libre}.
En el lenguaje no hay ninguna clase de cuantificación, a nivel de tipos, para estos elementos; pero nos referiremos de esta manera informal a todas las variables que ocurran dentro de uno.
El propósito de esta definición es simplificar la escritura de algunas reglas. %es poder facilitar el chequeo de las condiciones previas.
\begin{align*}
&\FTV{\T{int}}
&=&\;
\emptyset
\\
&\FTV{\T{real}}
&=&\;
\emptyset
\\
&\FTV{\T{bool}}
&=&\;
\emptyset
\\
&\FTV{\T{char}}
&=&\;
\emptyset
\\
&\FTV{\T{pointer} \; \theta}
&=&\;
\FTV{\theta}
\\
&\FTV{\T{array} \; as_1, \ldots, as_n \; \T{of} \; \theta}
&=&\;
\FTV{\theta}
\\
&\FTV{tv}
&=&\;
\{ tv \}
\\
&\FTV{tn}
&=&\;
\emptyset
\\
&\FTV{tn \; \T{of} \; \theta_1, \ldots, \theta_n}
&=&\;
\FTV{\theta_1} \cup \ldots \cup \FTV{\theta_n}
\end{align*}

\subsection{Chequeos para Declaración de Tipos}

%A continuación, precisaremos las reglas para analizar las declaraciones de tipo en un programa.
Con las mismas, buscamos verificar una serie de propiedades necesarias para asegurar la corrección de las construcciones aludidas.
En el chequeo de declaraciones de tipo se debe garantizar la unicidad de los identificadores empleados para representar determinadas componentes, como los nombres de constantes enumeradas, campos de tupla, y parámetros de tipo.
Adicionalmente, debemos probar la validez de los tipos especificados dentro de las declaraciones.
Incluso, hay que asegurar el uso adecuado de las variables de tipo introducidas como parámetros de las definiciones.

\subsubsection{Reglas para Declaración de Tipos}

%Definidas las reglas para chequear cuando un tipo es válido, y la función que calcula las variables de tipo que ocurren en el mismo, comenzaremos con la especificación de las reglas empleadas en la prueba de corrección para declaraciones de tipo.
%Cuando se determina que una definición esta \textit{bien formada}, su información es añadida al contexto apropiado y se continua con el análisis del programa.
%Notar que l
La prueba de una serie de declaraciones de tipo responde al orden en que las mismas se encuentran especificadas, esto implica, que las reglas no permiten la definición mutua entre estas declaraciones.
De esta forma, un tipo definido solo será accesible para las declaraciones posteriores al mismo.

El \textit{juicio de tipado} que prueba la validez de una declaración de tipo será aquel que tiene una derivación que prueba la correcta extensión del contexto original con la nueva declaración.
%Luego del análisis, se producirá un nuevo contexto donde se agrega la información de la definición recientemente verificada al contexto inicial.
Recordar que al realizarse estas extensiones, se deben seguir respetando las invariantes de consistencia para los conjuntos involucrados.
\begin{gather*}
\PI{T} \DASH{td} typedecl : \PI{T}'
\end{gather*}

La regla para la definición de tipos enumerados es simple, debido a las invariantes de los contextos de tipos. Notar que una extensión válida del contexto original ya nos asegura la unicidad del nombre de tipo respecto a las otras definiciones, y que los constructores empleados en la declaración no son utilizados en otras definiciones de tipo.
%Con la construcción del nuevo conjunto, uno puede asegurar la unicidad del nombre de tipo respecto a las otras definiciones, y que los constructores empleados en la declaración no son utilizados en otras definiciones de tipo.

\begin{DTRegla}
\label{DTEnumerado}
Enumerados
\begin{prooftree}
\AxiomC{}
\UnaryInfC
{$
\pi_{e}, \pi_{s}, \pi_{t} \DASH{td} \T{enum} \; tn = cn_1, \ldots, cn_m : \pi'_{e}, \pi_{s}, \pi_{t}
$}
\end{prooftree}
donde $\pi'_{e} = (tn, \{ cn_1, \ldots, cn_m \}) \triangleright \pi_{e}$.
\end{DTRegla}

Para los sinónimos se tiene que asegurar que el conjunto de parámetros de la declaración coincida con el conjunto de variables de tipo utilizadas en la definición del mismo.
Esta condición evita la ocurrencia de variables \textit{libres} en el cuerpo de la declaración, y también obliga el uso de todos los argumentos de la misma.
Además, el tipo que propiamente define al sinónimo tiene que ser válido.
Notar que no se permite la utilización de identificadores de tamaños dinámicos.

%La invariante del contexto garantiza la unicidad de los identificadores empleados como parámetros de la declaración.

\begin{DTRegla}
\label{DTSinonimo}
Sinónimos sin Argumentos
\begin{prooftree}
\AxiomC
{$
\pi_{e}, \pi_{s}, \pi_{t} \DASH[\emptyset_{tv}, \emptyset_{sn}]{t} \theta
$}
\UnaryInfC
{$
\pi_{e}, \pi_{s}, \pi_{t} \DASH{td} \T{syn} \; tn = \theta : \pi_{e}, \pi'_{s}, \pi_{t}
$}
\end{prooftree}
donde $\pi'_{s} = (tn, \emptyset, \theta) \triangleright \pi_{s}$.
\end{DTRegla}

\begin{DTRegla}
\label{DTSinonimoP}
Sinónimos con Argumentos
\begin{prooftree}
\AxiomC
{$
\pi_{tv} \subseteq \FTV{\theta}
$}
\AxiomC
{$
\pi_{e}, \pi_{s}, \pi_{t} \DASH[\pi_{tv}, \emptyset_{sn}]{t} \theta
$}
\BinaryInfC
{$
\pi_{e}, \pi_{s}, \pi_{t} \DASH{td} \T{syn} \; tn \; \T{of} \; tv_1, \dots, tv_l = \theta : \pi_{e}, \pi'_{s}, \pi_{t}
$}
\end{prooftree}
donde $\pi'_{s} = (tn, \{ tv_1, \ldots, tv_l \}, \theta) \triangleright \pi_{s}$, y $\pi_{tv} = \{ tv_1, \ldots, tv_l \}$.
\end{DTRegla}

Las verificaciones para tuplas son similares a las de sinónimos, salvo que se deben adecuar para los múltiples campos de la misma.
Hay que asegurar la igualdad entre los parámetros de la definición, y las variables de tipo que ocurren en todos los campos.
Adicionalmente, se tienen que analizar todos los tipos para asegurar su corrección.
Por último, se deben respetar las invariantes de unicidad tanto para los nombres de campos, como para los argumentos de tipo.

\begin{DTRegla}
\label{DTTupla}
Tuplas sin Argumentos
\begin{prooftree}
\AxiomC
{$
\pi_{e}, \pi_{s}, \pi_{t} \DASH[\emptyset_{tv}, \emptyset_{sn}]{t} \theta_i
$}
\UnaryInfC
{$
\pi_{e}, \pi_{s}, \pi_{t} \DASH{td} \T{tuple} \; tn = fn_1: \theta_1, \ldots, fn_m: \theta_m : \pi_{e}, \pi_{s}, \pi'_{t}
$}
\end{prooftree}
donde $\pi'_{t} = (tn, \emptyset, \{ fn_1: \theta_1, \ldots, fn_m: \theta_m \}) \triangleright \pi_{t}$.
\end{DTRegla}

\begin{DTRegla}
\label{DTTupleP}
Tuplas con Argumentos
\begin{prooftree}
\AxiomC
{$
\pi_{tv} \subseteq \FTV{\theta_1} \cup \ldots \cup \FTV{\theta_m}
$}
\AxiomC
{$
\pi_{e}, \pi_{s}, \pi_{t} \DASH[\pi_{tv}, \emptyset_{sn}]{t} \theta_i
$}
\BinaryInfC
{$
\pi_{e}, \pi_{s}, \pi_{t} \DASH{td} \T{tuple} \; tn \; \T{of} \; tv_1, \ldots, tv_l = fn_1: \theta_1, \ldots, fn_m: \theta_m : \pi_{e}, \pi_{s}, \pi'_{t}
$}
\end{prooftree}
donde $\pi'_{t} = (tn, \{ tv_1, \ldots, tv_l \}, \{ fn_1: \theta_1, \ldots, fn_m: \theta_m \}) \triangleright \pi_{t}$, y $\pi_{tv} = \{ tv_1, \ldots, tv_l \}$.
\end{DTRegla}

Las siguientes reglas son las que permite la definición de tipos recursivos, notar que esta es 
bastante restrictiva. La única posibilidad de declarar un tipo que se define en términos de si mismo es mediante el uso de punteros dentro de tuplas.
%Por lo tanto, tiene sentido que esta regla sea una variante de las reglas previas.
Solo se permiten utilizar las mismas variables de tipo paramétricas que en la definición, e incluso se las debe especificar en el mismo orden.
%En particular, si un tipo representa una llamada recursiva al tipo declarado, entonces el mismo quedará exceptuado del chequeo de validez para tipos.
%Las invariantes de consistencia para los contextos se deben respetar al igual que en las reglas anteriores.

\begin{DTRegla}
\label{DTRecursion}
Recursión para Tuplas sin Argumentos
\begin{prooftree}
\AxiomC
{$
\theta_i \neq \T{pointer} \; tn \implies \pi_{e}, \pi_{s}, \pi_{t} \DASH[\emptyset_{tv}, \emptyset_{sn}]{t} \theta_i
$}
\UnaryInfC
{$
\pi_{e}, \pi_{s}, \pi_{t} \DASH{td} \T{tuple} \; tn = fn_1: \theta_1, \ldots, fn_m: \theta_m : \pi_{e}, \pi_{s}, \pi'_{t}
$}
\end{prooftree}
donde $\pi'_{t} = (tn, \emptyset, \{ fn_1: \theta_1, \ldots, fn_m: \theta_m \}) \triangleright \pi_{t}$.
\end{DTRegla}

\begin{DTRegla}
\label{DTRecursionP}
Recursión para Tuplas con Argumentos
\begin{prooftree}
\AxiomC
{$
\pi_{tv} \subseteq \FTV{\theta_1} \cup \ldots \cup \FTV{\theta_m}
$}
\AxiomC
{$
\theta_i \neq \T{pointer} \; tn \; \T{of} \; tv_1, \ldots, tv_l \implies \pi_{e}, \pi_{s}, \pi_{t} \DASH[\pi_{tv}, \emptyset_{sn}]{t} \theta_i
$}
\BinaryInfC
{$
\pi_{e}, \pi_{s}, \pi_{t} \DASH{td} \T{tuple} \; tn \; \T{of} \, tv_1, \ldots, tv_l = fn_1: \theta_1, \ldots, fn_m: \theta_m : \pi_{e}, \pi_{s}, \pi'_{t}
$}
\end{prooftree}
donde $\pi'_{t} = (tn, \{ tv_1, \ldots, tv_l \}, \{ fn_1: \theta_1, \ldots, fn_m: \theta_m \}) \triangleright \pi_{t}$, y $\pi_{tv} = \{ tv_1, \ldots, tv_l \}$.
\end{DTRegla}

\subsubsection{Ejemplo de Prueba}

Siguiendo con el ejemplo especificado previamente, presentaremos la prueba de corrección para dos declaraciones de tipo particulares.
Primero verificaremos una estructura de tipo tupla, y luego realizaremos lo apropiado para un sinónimo de tipo.
Supongamos los siguientes contextos, los cuales serán los resultados obtenidos luego de realizados los respectivos análisis.
\begin{gather*}
\pi_{s} = \{ (list, \{ A \}, \T{pointer} \; node \; \T{of} \; A) \}
\\
\pi_{t} = \{ (node, \{ Z \}, \{ elem: Z, next: \T{pointer} \; node \; \T{of} \; Z \} ) \}
\end{gather*}

Comenzando con los contextos de tipos vacíos, se puede realizar la prueba para la estructura de tipo tupla \textit{nodo}.
Como se preciso anteriormente, debido que el campo \textit{next} realiza una llamada recursiva al tipo que estamos definiendo, se omite la verificación de su tipo.
Notar el uso de las reglas para las variables de tipo, y la recursión en declaración de tuplas.

\begin{Prueba}
\label{PDTNode}
Derivación de corrección para la declaración de tipo \emph{node}.
\begin{prooftree}
\AxiomC
{$
\{ Z \} \subseteq \FTV{Z} \cup \FTV{\T{pointer} \; node \; \T{of} \; Z}
$}
\AxiomC
{$
Z \in \{ Z \}_{tv}
$}
\RightLabel{\RULE{\ref{TVariable}}}
\UnaryInfC
{$
\emptyset_{e}, \emptyset_{s}, \emptyset_{t} \DASH[\{ Z \}_{tv}, \emptyset_{sn}]{t} Z
$}
\RightLabel{\RULE{\ref{DTRecursionP}}}
\BinaryInfC
{$
\emptyset_{e}, \emptyset_{s}, \emptyset_{t} \DASH{td} \T{tuple} \; node \; \T{of} \; Z = elem : Z, next : \T{pointer} \; node \; \T{of} \; Z : \emptyset_{e}, \emptyset_{s}, \pi_{t}
$}
\end{prooftree}
\end{Prueba}

Utilizando el contexto de tuplas obtenido en la derivación anterior, se puede realizar la prueba del sinónimo de tipo \textit{lista}.
La demostración de corrección del tipo que define al sinónimo ya fue precisada en la sección previa.
Por lo tanto, aplicando la regla para sinónimos con parámetros, junto con la demostración mencionada, podemos construir la derivación del \textit{juicio de tipado} correspondiente a la declaración.

\begin{Prueba}
\label{PDTList}
Derivación de corrección para la declaración de tipo \emph{lista}.
\begin{prooftree}
\AxiomC
{$
\{ A \} \subseteq \FTV{\T{pointer} \; node \; \T{of} \; A}
$}
\AxiomC{Prueba~\ref{PTPointerNode}}
\RightLabel{\RULE{\ref{TPuntero}}}
\UnaryInfC
{$
\emptyset_{e}, \emptyset_{s}, \pi_{t} \DASH[\{ A \}_{tv}, \emptyset_{sn}]{t} \T{pointer} \; node \; \T{of} \; A
$}
\RightLabel{\RULE{\ref{DTSinonimoP}}}
\BinaryInfC
{$
\emptyset_{e}, \emptyset_{s}, \pi_{t} \DASH{td} \T{syn} \; list \; \T{of} \; A = \T{pointer} \; node \; \T{of} \; A : \emptyset_{e}, \pi_{s}, \pi_{t}
$}
\end{prooftree}
\end{Prueba}

Notar que empleando las distintas reglas introducidas a lo largo del capítulo, somos capaces de probar la validez de la declaración para la \textit{lista abstracta} en el lenguaje.
El fragmento de código especificado~(\ref{listaAbstracta}) comprende la implementación, utilizando la sintaxis concreta de \Lenguaje{}, para las definiciones de tipo recientemente analizadas.

\begin{lstlisting}[ style = lang, caption = Implementación de Lista Abstracta, label = listaAbstracta ]
type node of (Z) = tuple
                     elem : Z,
                     next : pointer of node of (Z)
                   end tuple

type list of (A) = pointer of node of (A)
\end{lstlisting}

% \begin{adjustbox}{center, minipage = {\paperwidth}, margin = 0ex 1.5ex}