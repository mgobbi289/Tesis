% Chequeos para Programas
\subsection{Chequeos para Programas}

Finalmente para concluir con las reglas de derivación de los \textit{juicios de tipado}, solo resta formalizar el juicio y la regla para el chequeo de un programa.
El siguiente \textit{juicio de tipado} determina que el chequeo de tipos de un programa es válido bajo los contextos $\PI{T}$ y $\PI{FP}$.
% IDEA: El análisis puede ser separado en dos clases de tareas distintas.
% IDEA: Una donde se verifican las declaraciones de tipo, y otra donde se examinan las funciones y los procedimientos.
% IDEA: En el caso hipotético donde se quiera probar la corrección de un programa cerrado, claramente se deberá realizar el análisis sin ninguna clase de información contextual inicial.
\begin{gather*}
\PI{T}, \PI{FP} \DASH{p} td_{1} \ldots td_{n} \quad fpd_{1} \ldots fpd_{m}
\end{gather*}

\iffalse
\subsubsection{Regla para Programas}

El \textit{juicio de tipado} determina que un programa particular es válido estáticamente, de acuerdo a los contextos comprendidos por $\PI{T}$ y $\PI{FP}$.
La información contenida en estos contextos no necesariamente debe ser nula.
Asumiendo que sus definiciones son correctas, los contextos pueden ser inicializados de manera arbitraria.
Donde la noción de corrección estará sujeta a la satisfacción de las propiedades especificadas a lo largo del informe.
\fi

La única regla asociada a este \textit{juicio} tiene como premisas a las derivaciones de juicios para las declaraciones de tipo y las declaraciones de funciones y procedimientos, que conforman al programa.
% IDEA: La regla correspondiente al \textit{juicio} para programas necesita verificar todas las definiciones de tipo, y las declaraciones de funciones y procedimientos, que lo conforman.
Notar que los contextos involucrados en el análisis se construyen de manera incremental, lo cual manifiesta la relevancia del orden de las declaraciones que aparecen en el programa.

\begin{PRegla}
\label{PPrograma}
Programas
\begin{prooftree}
\AxiomC
{$
\PI{T}^{i-1} \DASH{td} td_{i} : \PI{T}^{i}
$}
\AxiomC
{$
\PI{T}^{n}, \PI{FP}^{j-1} \DASH{fp} fpd_{j} : \PI{FP}^{j}
$}
\BinaryInfC
{$
\PI{T}^{0}, \PI{FP}^{0} \DASH{p} td_{1} \ldots td_{n} \quad fpd_{1} \ldots fpd_{m}
$}
\end{prooftree}
% IDEA: ...donde en un programa cerrado, los contextos iniciales son vacíos  $\PI{T}^{0} = \PI{FP}^{0} = \emptyset$.
\end{PRegla}

\section{Más Chequeos Estáticos}

% IDEA: Durante el análisis estático, se verifica una colección de propiedades con la intención de garantizar cierto grado de corrección sobre el programa analizado.
% IDEA: Claramente, existen ciertas propiedades deseables que no pueden ser verificadas solo con la aplicación de las reglas para \textit{juicios de tipado}.
% IDEA: En la sección actual nos concentraremos en la formalización de las propiedades estáticas que aún restan por definir en el análisis.

Los \textit{juicios de tipado} junto con sus reglas, conforman una parte importante de los chequeos estáticos.
En esta sección definiremos formalmente otros chequeos estáticos, los cuales pueden ser pensados como predicados sobre programas que aseguran la corrección de más propiedades sobre estos.
En una función, quisiéramos evitar que en el cuerpo se asignen los parámetros de su prototipo, mientras que para el caso del retorno es esperable que ocurra.
En un procedimiento, los parámetros denotados como lectura no deben ser asignados, y viceversa, los parámetros denotados como escritura no pueden ser utilizados como valores.
% IDEA: Además, los identificadores de tamaños dinámicos deben ser utilizados como constantes, y por lo tanto no deberían ser asignables.
% IDEA: Adicionalmente el prototipo también introduce elementos polimórficos, como los tamaños dinámicos, que presentan sus características propias.
% IDEA: De acuerdo a los elementos mencionados, será necesario asegurar la correcta lectura y/o escritura de cada uno en base a su naturaleza específica.
% IDEA: En la definición de las propiedades que se pretenden verificar, realizaremos la suposición que el lenguaje utiliza \textit{pasaje por referencia}.

\subsection{Funciones para Variables}

% IDEA: Para formalizar de manera precisa las propiedades a verificar mencionadas, es necesario definir una serie de funciones auxiliares adicionales.
Para formalizar estas propiedades sobre funciones y procedimientos del programa, definiremos una serie de funciones auxiliares adicionales.
Su utilidad radicará en que permiten distinguir el uso que se hace de los identificadores de variables en el cuerpo de las funciones, y los procedimientos.
% IDEA: Recordar que las variables, los parámetros, y los tamaños dinámicos, emplean la misma categoría de identificadores en la sintaxis concreta del lenguaje.
% IDEA: De manera informal, haremos referencia a todos estos identificadores de la misma forma sin hacer distinciones.

\subsubsection{Variables Libres}

% Referencia al Calculo Lambda...
En el cálculo lambda, se denomina \textit{libre} a toda variable que no se encuentra cuantificada.
En nuestro caso, el lenguaje no provee ninguna especie de cuantificación a nivel de expresiones, por lo que toda variable en una expresión es una \textit{variable libre}.
% IDEA: En un estudio posterior, para dar semántica a estas construcciones del lenguaje se utilizará el concepto de \textit{estado}; entendido como una función que asigna valores a las variables.
La siguiente función calcula el conjunto de \textit{variables libres} que ocurren en una expresión determinada.
Notar que es posible aplicar un razonamiento análogo para calcular las ocurrencias \textit{libres} en las sentencias del lenguaje, pero no nos será necesario.
\begin{gather*}
FV^{\NT{expression}}: \NT{expression} \rightarrow \{ \NT{id} \}
\end{gather*}

% IDEA: El comportamiento de la función es trivial.
% IDEA: Toda ocurrencia de una variable, es una ocurrencia \textit{libre}.
% IDEA: Se debe propagar la aplicación para las expresiones internas, reuniendo los identificadores de variables encontrados.
Notar que ciertos identificadores, como los nombres de funciones, no serán considerados por la función ya que estos no formarán parte de los chequeos que definiremos.
% IDEA: El valor asociado a los identificadores de estas construcciones no depende del \textit{estado} del programa, lo cual implica que no cambia de manera dinámica.
% IDEA: Siendo más precisos, su toma de valor es estática de acuerdo a la declaración asociada en el código del programa.
% IDEA: El concepto de \textit{entorno} es utilizado en este contexto.
% IDEA: Por lo tanto no corresponde considerar estos identificados como ocurrencias \textit{libres}.
Por cuestiones de claridad, los supraíndices de las funciones definidas serán omitidos.
\begin{alignat*}{2}
&\FV{ct}
&&\;=\;
\emptyset
\\
&\FV{f(e_1, \ldots, e_n)}
&&\;=\;
\FV{e_1} \cup \ldots \cup \FV{e_n}
\\
&\FV{e_1 \oplus e_2}
&&\;=\;
\FV{e_1} \cup \FV{e_2}
\\
&\FV{\ominus e}
&&\;=\;
\FV{e}
\\
&\FV{x}
&&\;=\;
\{ x \}
\\
&\FV{v [e_1, \ldots, e_n]}
&&\;=\;
\FV{v} \cup \FV{e_1} \cup \ldots \cup \FV{e_n}
\\
&\FV{v.fn}
&&\;=\;
\FV{v}
\\
&\FV{\star v}
&&\;=\;
\FV{v}
\end{alignat*}

\subsubsection{Variables de Escritura}

Una variable es considerada de \textit{escritura}, cuando su valor será potencialmente modificado durante la ejecución de una sentencia.
Las siguientes funciones calculan el conjunto de \textit{variables de escritura} que ocurren en expresiones o sentencias, respectivamente.
Debido que necesitamos información sobre las etiquetas declaradas para cada parámetro introducido en el prototipo de los procedimientos, tenemos que indexar la definición de la función en el contexto que contiene a las declaraciones de procedimientos.
Notar que esta propiedad a definir (junto con las que vendrán) asume que el programa está bien tipado; es decir, podemos construir una derivación para un determinado juicio de tipado (y por lo tanto, construir el contexto que necesitamos).
\begin{gather*}
WV^{\NT{expression}}: \NT{expression} \rightarrow \{ \NT{id} \}
\qquad
WV^{\NT{sentence}}_{\pi_{p}}: \NT{sentence} \rightarrow \{ \NT{id} \}
\end{gather*}

% IDEA: Iniciaremos con la formalización de la función para el caso de las sentencias, ya que resulta ser el más intuitivo.
% IDEA: Existen varias sentencias que pueden modificar el estado de una variable.
Las sentencias que pueden modificar el estado de una variable son la asignación, el procedimiento nativo \textit{alloc}, y la llamada de procedimientos.
% IDEA: Claramente no todas las ocurrencias de una variable serán de \textit{escritura}, por lo que es necesario propagar la aplicación de la función sobre las expresiones especificadas dentro de las sentencias mencionadas.
% IDEA: El razonamiento detrás de la definición para la función de expresiones, estará condicionado por esta particularidad.
Notar el abuso de notación al aplicar la función a un bloque de sentencias $ss$, en esta situación el conjunto de \textit{variables de escritura} se obtiene al acumular la aplicación sobre cada una de las sentencias que lo conforman.
Supongamos que la condición $(p, \{ io_1 \; a_1: \theta_1, \ldots, io_n \; a_n: \theta_n \}, cs) \in \pi_{p}$ es válida.
\begin{alignat*}{2}
&\WVs{\T{skip}}
&&\;=\;
\emptyset
\\
&\WVs{v := e}
&&\;=\;
\WV{v}
\\
&\WVs{\T{alloc} \; v}
&&\;=\;
\WV{v}
\\
&\WVs{\T{free} \; v}
&&\;=\;
\emptyset
\\
&\WVs{\T{if} \; e \; \T{then} \; ss \; \T{else} \; ss'}
&&\;=\;
\WVs{ss} \cup \WVs{ss'}
\\
&\WVs{\T{while} \; e \; \T{do} \; ss}
&&\;=\;
\WVs{ss}
\\
&\WVs{\T{for} \; x := \, e_1 \; \T{to} \; e_2 \; \T{do} \; ss}
&&\;=\;
\WVs{ss} - \{ x \}
\\
&\WVs{\T{for} \; x := \, e_1 \; \T{downto} \; e_2 \; \T{do} \; ss}
&&\;=\;
\WVs{ss} - \{ x \}
\\
&\WVs{p(e_1, \ldots, e_n)}
&&\;=\;
\{ \WV{e_i} \mid io_i = \T{out} \vee io_i = \T{in/out} \}
\end{alignat*}

% IDEA: Existe una particularidad en la definición de la función para la sentencia \textit{for}.
% IDEA: La variable $x$ es declarada, inicializada, y modificada, de acuerdo al rango especificado en la sentencia.
La definición para el caso de la sentencia \textit{for} es interesante; la variable $x$ es declarada y actualizada de forma implícita dentro del cuerpo de la sentencia.
Por lo que esta variable quedará fuera del conjunto de variables que deseamos considerar como asignables.
% IDEA: Puesto que el alcance de la variable solo comprende al cuerpo, carecería de sentido considerar una ocurrencia de \textit{escritura} dentro del cuerpo, como una ocurrencia de \textit{escritura} en la sentencia general.

% IDEA: Debido que la llamada a un procedimiento toma una serie de expresiones como argumentos, es necesario realizar un análisis detallado para el cálculo de las \textit{variables de escritura}.
Otro caso interesante de la definición es la llamada a procedimientos, ya que se aplica a una serie de expresiones que podrían llegar a ser modificadas en caso de corresponderse con un parámetro del procedimiento etiquetado con $\T{out}$ o $\T{in/out}$.
% IDEA: Las etiquetas de \textit{entrada/salida} cumplen un rol fundamental en este aspecto.
% IDEA: Las expresiones que correspondan a un parámetro de salida en la definición del procedimiento, serán potencialmente modificadas.
En cambio, el resto de los argumentos que correspondan a un parámetro de entrada no serán modificados por la llamada.
% IDEA: Idealmente, las mismas deberían ser variables.

Técnicamente ninguna ocurrencia de una variable en una expresión es de \textit{escritura}, ya que su evaluación nunca debería producir efectos secundarios.
En virtud que una expresión puede suceder dentro de una sentencia, la cual tiene el potencial de modificar el estado de las variables que ocurren en ella, la función tendrá el siguiente comportamiento.
Notar que en soledad la definición no tiene utilidad, solo posee sentido al complementar la función para sentencias.
\begin{alignat*}{3}
&\WV{e}
&&\;=\;
\emptyset
&&\;e \notin \NT{variable}
\\
&\WV{x}
&&\;=\;
\{ x \}
\\
&\WV{v [e_1, \ldots, e_n]}
&&\;=\;
\WV{v}
\\
&\WV{v.fn}
&&\;=\;
\WV{v}
\\
&\WV{\star v}
&&\;=\;
\emptyset
\end{alignat*}

Es claro que no es posible modificar el valor de una expresión, la cual no representa a una variable, por lo que el comportamiento de la función resulta trivial en este aspecto.
Para los operadores de variables, en el acceso de arreglos o tuplas, se debe propagar la función hasta obtener el identificador de la variable inicial.
En el caso de los punteros, hay un detalle particular que es importante aclarar sobre el \textit{aliasing}.

El \textit{aliasing} es una situación que ocurre cuando la misma posición de memoria se puede acceder utilizando distintos identificadores.
En nuestro caso, se da cuando hay más de un puntero diferente que referencia a la misma estructura en memoria.
Debido a esta situación, resulta complejo analizar estáticamente cuando el elemento referido por un puntero podrá ser potencialmente modificado.
Por lo tanto, esta condición implica que la definición de la propiedad que pretendemos dar probablemente no sea completa.
% IDEA: En consecuencia, la tarea de asegurar la correcta modificación de posiciones de memoria reservadas por el usuario, es relegada al análisis dinámico.
% IDEA: Por lo tanto, es posible que de acuerdo a las verificaciones estáticas especificadas en la sección, sea necesario diseñar chequeos análogos para efectuar durante la ejecución del programa.
% IDEA: Puntualmente, evitar la modificación de la memoria reservada para punteros, pasados como argumentos de funciones o entradas de procedimientos.

\subsubsection{Variables de Lectura}

Una variable es considerada de \textit{lectura} cuando su valor es potencialmente utilizado en la evaluación de una expresión durante la ejecución de una sentencia.
Al igual que ocurría con el cálculo de las \textit{variables de escritura}, acá también es necesario asumir el correcto tipado del programa; particularmente de sus procedimientos.
% IDEA: La siguiente función calculará el conjunto de \textit{variables de lectura} que ocurren en una expresión o sentencia determinada.
% IDEA: Debido que la información sobre las etiquetas de los procedimientos declarados es necesaria para el cómputo de la función, en el último caso se indexa con el contexto correspondiente.
\begin{gather*}
RV^{\NT{expression}}: \NT{expression} \rightarrow \{ \NT{id} \}
\qquad
RV^{\NT{sentence}}_{\pi_{p}}: \NT{sentence} \rightarrow \{ \NT{id} \}
\end{gather*}

Es interesante notar que el conjunto de \textit{variables de lectura} de una sentencia, no necesariamente será disjunto al correspondiente conjunto de \textit{variables de escritura}.
Un ejemplo concreto podría ser la asignación de una variable a si misma $x := x$, donde la primer ocurrencia es de \textit{escritura} y la segunda ocurrencia es de \textit{lectura}.

\iffalse
Comenzaremos con la formalización de la función para el caso de las sentencias, ya que resulta ser el más intuitivo.
Recordar que las sentencias que pueden modificar el estado de una variable comprenden la asignación, los procedimientos \textit{alloc} y \textit{free}, y los procedimientos declarados por el usuario.
Resulta claro que incluso en estas situaciones, ciertas ocurrencias de variables podrán ser de \textit{lectura}.
El razonamiento detrás de la definición para la función de expresiones, estará condicionado por esta particularidad.
En cambio para el resto de las sentencias, las ocurrencias de variables solo serán de \textit{lectura}; donde la función para \textit{variables libres} de expresiones es fundamental en este contexto.
Nuevamente notar el abuso de notación cuando se aplica la función a un bloque de sentencias $ss$, en esta situación se procede de manera análoga al caso anterior.
\fi

La siguiente declaración no será una definición dual a la función previa, principalmente por dos razones.
La primera es que cualquier \textit{variable libre} que ocurra en un sentencia, podrá ser considera de lectura siempre y cuando no aparezca exclusivamente en la componente izquierda de una asignación.
La segunda es que, inclusive en ese caso, podría ocurrir que exista una variable de lectura, en el lado izquierdo de la asignación, cuando es utilizada como un valor para referirse a alguna componente de una estructura más compleja.
Por ejemplo, el acceso a una posición de memoria.
% IDEA: Incluso puede suceder que en una expresión, distintas ocurrencias de una misma variable cumplan diferentes roles.
% IDEA: De todas maneras, es posible reconocer fácilmente que existe una analogía entre ambas funciones.
Supongamos que la condición $(p, \{ io_1 \; a_1: \theta_1, \ldots, io_n \; a_n: \theta_n \}, cs) \in \pi_{p}$ es válida.
\begin{alignat*}{2}
&\RVs{\T{skip}}
&&\;=\;
\emptyset
\\
&\RVs{v := e}
&&\;=\;
\RV{v} \cup \FV{e}
\\
&\RVs{\T{alloc} \; v}
&&\;=\;
\RV{v}
\\
&\RVs{\T{free} \; v}
&&\;=\;
\FV{v}
\\
&\RVs{\T{if} \; e \; \T{then} \; ss \; \T{else} \; ss'}
&&\;=\;
\FV{e} \cup \RVs{ss} \cup \RVs{ss'}
\\
&\RVs{\T{while} \; e \; \T{do} \; ss}
&&\;=\;
\FV{e} \cup \RVs{ss}
\\
&\RVs{\T{for} \; x := \, e_1 \; \T{to} \; e_2 \; \T{do} \; ss}
&&\;=\;
\FV{e_1} \cup \FV{e_2} \cup (\RVs{ss} - \{ x \})
\\
&\RVs{\T{for} \; x := \, e_1 \; \T{downto} \; e_2 \; \T{do} \; ss}
&&\;=\;
\FV{e_1} \cup \FV{e_2} \cup (\RVs{ss} - \{ x \})
\\
&\RVs{p(e_1, \ldots, e_n)}
&&\;=\;
\{ \FV{e_i} \mid io_i = \T{in} \vee io_i = \T{in/out} \}
\cup \ldots
\\&&&\qquad
\ldots \cup
\{ \RV{e_i} \mid io_i = \T{out} \}
\end{alignat*}

De forma similar a la definición previa para el caso de la llamada a un procedimiento, si una expresión corresponde a un parámetro de entrada, entonces todas las ocurrencias de variables serán de \textit{lectura}.
Adicionalmente podría ocurrir una situación similar a la que anteriormente mencionamos con la asignación; una expresión que es utilizada como argumento para un parámetro que fue etiquetado con $\T{out}$, podría contener \textit{variables de lectura}.
% IDEA: En cambio, cuando una expresión coincide con un parámetro de salida, será necesario realizar un análisis meticuloso para determinar cual es el uso que se hace de las distintas ocurrencias de variables dentro de la expresión.

Claramente toda ocurrencia de variable en una expresión es de \textit{lectura}, ya que su valor será potencialmente utilizado en la evaluación de la misma.
Sin embargo la siguiente definición complementa la definición para sentencias, por lo que de forma aislada puede resultar contra intuitiva.
En este caso es posible pensar a la función para \textit{variables de lectura} como una versión dual de la función para \textit{variables de escritura}.
% IDEA: Es posible pensar a la función para \textit{variables de lectura}, como opuesta a la función para \textit{variables de escritura}.
% IDEA: A causa que una expresión puede suceder dentro de una sentencia, la cual posee el potencial de modificar el estado de las variables que ocurren en ella, la función presenta ciertas particularidades que pueden resultar contra intuitivas.
% IDEA: Notar que en soledad la definición carece de sentido, y solo tomará significado al completar la función para sentencias.
\begin{alignat*}{3}
&\RV{e}
&&\;=\;
\FV{e}
&&\;e \notin \NT{variable}
\\
&\RV{x}
&&\;=\;
\emptyset
\\
&\RV{v [e_1, \ldots, e_n]}
&&\;=\;
\RV{v} \cup \FV{e_1} \cup \ldots \cup \FV{e_n}
\\
&\RV{v.fn}
&&\;=\;
\RV{v}
\\
&\RV{\star v}
&&\;=\;
\FV{v}
\end{alignat*}

Cuando una expresión no es una variable entonces solo podrá ser evaluada, lo cual implica que todas sus ocurrencias de variables serán de \textit{lectura}.
Para los operadores de variables, solamente se consideran a las expresiones empleadas para acceder a los diferentes elementos que componen a la estructura de la variable.
En el caso de los punteros, todas las variables que intervienen en el acceso a memoria son consideradas de \textit{lectura}.
% IDEA: Recordar que a causa del problema del \textit{aliasing}, la verificación del uso apropiado de la memoria es delegada al análisis dinámico.

\subsection{Predicados sobre Funciones y Procedimientos}

Diremos que un programa es válido estáticamente cuando existe una derivación para su \textit{juicio de tipado}, y se satisfacen las propiedades sobre \textit{lectura/escritura} de variables para todas sus declaraciones de funciones y procedimientos.
Recordar que los contextos resultados del chequeo son importantes para la verificación de las propiedades.
% IDEA: En la formalización, haremos uso de los últimos conceptos definidos.
% IDEA: Notar que los contextos que intervienen en los predicados son construidos durante la derivación del \textit{juicio de tipado} correspondiente a la construcción analizada.

Extenderemos las funciones previamente definidas para que puedan ser aplicadas al cuerpo de una función o de un procedimiento.
De esta manera se logra flexibilizar la notación, lo cual facilitará la lectura de los predicados.
La aplicación de las funciones debe ser propagada solamente a las sentencias que conforman al cuerpo, acumulando los resultados obtenidos.
\begin{align*}
\WVs{vd_1 \ldots vd_n \; s_1 \ldots s_m} &= \WVs{s_1} \cup \ldots \cup \WVs{s_m}
\\
\RVs{vd_1 \ldots vd_n \; s_1 \ldots s_m} &= \RVs{s_1} \cup \ldots \cup \RV{s_m}
\end{align*}

\subsubsection{Predicados para Funciones}

Supongamos tenemos la siguiente definición de función.
\begin{gather*}
\T{fun} \; f \; (a_1: \theta_1, \ldots, a_l: \theta_l) \; \T{ret} \; a_r: \theta_r \; \T{where} \; cs \; \T{in} \; body
\end{gather*}

Puesto que una función siempre devuelve un valor, no sería posible retornar algún resultado si la variable correspondiente no fuese asignada.
La variable de retorno $a_r$ debe ser modificada por las sentencias de la función.
Hay que tener en cuenta que estáticamente no es posible verificar si todas las posiciones de memoria alcanzables por la variable fueron debidamente inicializadas en el cuerpo.
Lo cual implica que la verificación no es completa, en el sentido que podría ser satisfecha y de todas maneras el retorno no ser asignado.
Además es necesario verificar que ninguno de los argumentos $a_i$ sea modificado por las sentencias que conforman al cuerpo.
% IDEA: Durante la ejecución del programa, una verificación dinámica podría encargarse de garantizar la completa asignación de la variable.

\begin{Predicado}
\label{PFRetorno}
Retorno de Función
\begin{gather*}
a_r \in \WVs{body}
\end{gather*}
\end{Predicado}

% IDEA: Una función no debería producir efectos secundarios en el contexto donde fue invocada.
% IDEA: Si el lenguaje utilizara \textit{pasaje por valor}, el predicado podría ser omitido debido que la función solo recibiría una copia del valor de sus argumentos.

\begin{Predicado}
\label{PFArgumento}
Argumento de Función
\begin{gather*}
\{ a_1, \ldots, a_l \} \cap \WVs{body} = \emptyset
\end{gather*}
\end{Predicado}

\subsubsection{Predicados para Procedimientos}

Supongamos tenemos la siguiente definición de procedimiento.
\begin{gather*}
\T{proc} \; p \; (io_1 \; a_1: \theta_1, \ldots, io_l \; a_l: \theta_l) \; \T{where} \; cs \; \T{in} \; body
\end{gather*}

Es necesario asegurar el correcto uso de los parámetros del procedimiento, de acuerdo a la etiqueta de \textit{entrada/salida} con la que fueron especificados.
Un parámetro definido como \textbf{in}, solo podrá ser utilizado como una variable de lectura.
Un parámetro definido como \textbf{out}, solo podrá ser utilizado como una variable de escritura.
% IDEA: En ambos casos, para evitar comportamientos inesperados durante la ejecución del programa, es preciso prevenir cualquier uso contrario al contemplado por la respectiva etiqueta.
Por supuesto, un parámetro declarado como \textbf{in/out}, podrá ser utilizado como una variable que cumple ambos roles y no es necesario realizar ninguna verificación al respecto.

\begin{Predicado}
\label{PPEntrada}
Entrada de Procedimiento
\begin{gather*}
\{ a_i \mid io_i = \T{in} \} \cap \WVs{body} = \emptyset
\end{gather*}
\end{Predicado}

\begin{Predicado}
\label{PPSalida}
Salida de Procedimiento
\begin{gather*}
\{ a_i \mid io_i = \T{out} \} \cap \RVs{body} = \emptyset
\end{gather*}
\end{Predicado}

\iffalse
\subsubsection{Predicados para Tamaños Dinámicos}

Supongamos que nos encontramos analizando una función o un procedimiento con alguna de las formas previas; donde el conjunto de tamaños dinámicos introducidos en el respectivo prototipo se define de la manera habitual.
\begin{gather*}
\pi_{sn} = \DAS{\theta_1} \cup \ldots \cup \DAS{\theta_l}
\end{gather*}

El tamaño de un arreglo es constante, lo cual implica que su valor no cambiará durante la ejecución de un programa.
En el caso particular de un tamaño dinámico, su valor es desconocido hasta que comienza la ejecución, momento donde queda determinado de manera única y definitiva.
Recordar que las variables, los parámetros de funciones y procedimientos, y los tamaños dinámicos, emplean la misma clase de identificadores; por lo cual, en cualquier sitio donde se espera alguno de estos elementos sintácticos, es posible especificar a cualquier otro.
En consecuencia, es necesario evitar la modificación de los tamaños dinámicos de arreglos durante la ejecución de la función o el procedimiento.

\begin{Predicado}
\label{PTDinamico}
Tamaño Dinámico
\begin{gather*}
\pi_{sn} \cap \WVs{body} = \emptyset
\end{gather*}
\end{Predicado}
\fi

\subsubsection{Ejemplo de Predicado}

Los predicados definidos pueden ser aplicados a las funciones y los procedimientos que fueron presentados a lo largo de este trabajo.
Para varios de estos ejemplos ya probamos que están bien tipados, por lo tanto nos limitaremos a calcular sus conjuntos de \textit{variables de lectura} y \textit{variables de escritura}.
% IDEA: La demostración formal sobre la satisfacción de los predicados, es dejada como ejercicio al lector.
Para facilitar la comprensión de los ejemplos, utilizaremos la notación $body_{fp}$ para representar al cuerpo de la función o el procedimiento $fp$.

Comenzando con la implementación del algoritmo de ordenación, observar que el arreglo a ordenar $a$ es empleado como una variable de \textit{lectura} y \textit{escritura} por ambos procedimientos.
En el fragmento~(\ref{swap}), se emplean los parámetros de entrada $i$ y $j$ de manera apropiada.
En el código~(\ref{selectionSort}), ninguna de las variables declaradas por las sentencias \textit{for} es considerada en los conjuntos obtenidos.
\begin{align*}
\WVs{body_{swap}} &= \{ temp, a \}
&
\WVs{body_{selectionSort}} &= \{ minP, a \}
\\
\RVs{body_{swap}} &= \{ temp, a, i, j \}
&
\RVs{body_{selectionSort}} &= \{ minP, a \}
\end{align*}
% Predicado - Entrada Procedimiento
% Predicado - Tamaños Dinámicos !!!

Pasando a las operaciones sobre \textit{listas abstractas}, el análisis es directo.
En el fragmento~(\ref{isEmpty}), el retorno $b$ es una variable de \textit{escritura}, mientras que el argumento $l$ es una variable de \textit{lectura}.
En el código~(\ref{tail}), el parámetro $l$ cumple ambos roles, y se puede notar que la variable $p$ declarada en el cuerpo solo es utilizada para su inicialización y subsiguiente liberación.
\begin{align*}
\WVs{body_{isEmpy}} &= \{ b \}
&
\WVs{body_{tail}} &= \{ l, p \}
\\
\RVs{body_{isEmpty}} &= \{ l \}
&
\RVs{body_{tail}} &= \{ l \}
\end{align*}
% Predicado - Argumentos/Retorno Función
% Predicado - ... Procedimiento (NO Lectura de Variable)

% WARNING
% Todas las variables, tamaños dinámicos, y parámetros, deben ser empleados.
% (Tamaños Dinámicos pueden NO ser empleados)
% WARNING
% Asegurar la lectura y escritura de todas las variables del programa.
% (Variables pueden NO ser de lectura)

% DINÁMICO
% Asignar valor en variables, antes de acceder a la misma.
% DINÁMICO
% Asignación completa al retorno de una función.
% DINÁMICO
% Evitar modificación punteros, ya sea argumento o entrada.

% WARNING
% Argumentos deben ser efectivamente leídos.

% PROPIEDAD
% Se permiten modificar memoria a través de punteros | Argumentos.

% WARNING
% Entradas IN deben ser efectivamente leídas.
% WARNING
% Entradas OUT deben ser efectivamente escritas.
% WARNING
% Entradas IN / OUT deben ser efectivamente leídas y escritas.

% PROPIEDAD
% Se permiten modificar memoria a través de punteros | IN.
% PROPIEDAD
% Solo puedo realizar free, alloc, y asignación a punteros | OUT.
% PROPIEDAD
% Se puede modificar libremente punteros | IN/OUT.