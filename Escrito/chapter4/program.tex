% Chequeos para Programas
\subsection{Chequeos para Programas}

Para finalizar con este capítulo, solo resta dar la regla para analizar un programa.
En resumen, para asegurar que un programa es válido (estáticamente), se deben realizar dos etapas de análisis.
En la primera se tienen que verificar progresivamente todas las declaraciones de tipo en el mismo, acumulando la información obtenida en el contexto de tipos definidos.
Luego como segunda etapa, se deben chequear una por una, todas las funciones y procedimientos del programa, almacenando la información conseguida en estos en los contextos adecuados.
Inicialmente, se debe comenzar con ambos contextos generales vacíos, y a medida que progrese el análisis, los mismos se irán expandiendo.

\begin{PRegla}
\label{PPrograma}
Programas
\begin{prooftree}
\AxiomC
{$
\PI{T}^{i-1} \DASH{td} td_{i} : \PI{T}^{i}
$}
\AxiomC
{$
\PI{T}^{n}, \PI{FP}^{j-1} \DASH{fp} fpd_{j} : \PI{FP}^{j}
$}
\BinaryInfC
{$
\PI{T}^{0}, \PI{FP}^{0} \DASH{p} td_{1} \ldots td_{n} \quad fpd_{1} \ldots fpd_{m}
$}
\end{prooftree}
donde los contextos iniciales son vacíos  $\PI{T}^{0} = \PI{FP}^{0} = \emptyset$.
\end{PRegla}