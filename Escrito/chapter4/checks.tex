% Chequeos
En este capítulo, definiremos formalmente los distintos chequeos estáticos del lenguaje y presentaremos algunos aspectos interesantes sobre su primera implementación.
Los fundamentos teóricos utilizados en esta sección están basados en \textit{Theories of Programming Languages} de \textit{Reynolds}~\cite{Reynolds}.
Siendo precisos los capítulos sobre el sistema de tipos (\texttt{15}), el subtipado (\texttt{16}), y el polimorfismo (\texttt{18}) resultaron indispensables para el desarrollo del informe.
Por el otro lado, la implementación de los chequeos estáticos siguió la idea planteada en el artículo de \textit{Castegren} y \textit{Reyes}~\cite{MonadicTC}.
Las secciones más relevantes para la realización de nuestro trabajo comprenden la definición del \textit{typechecker} (\texttt{2}), el soporte para \textit{backtraces} (\texttt{4}), y el agregado de \textit{warnings} (\texttt{5}).

\section{Metavariables}

% IDEA: Como mencionamos en capítulos previos, para el estudio del lenguaje nos apoyaremos en la especificación de su \textit{sintaxis abstracta}.
A lo largo del capítulo, se utilizarán diversas metavariables (a veces acompañadas de supraíndices, o subíndices) para representar distintas clases de construcciones sintácticas.
A continuación se listará cada una junto con el elemento sintáctico que comúnmente simbolizará, a menos que se indique lo contrario en el momento oportuno.

\begin{align*}
&\text{\underline{Expresiones}}
\\
&e       & &\NT{expression}     &  &ct      & &\NT{constant}       \\
&v       & &\NT{variable}       &  &n       & &\NT{integer}        \\
&x, a    & &\NT{id}             &  &r       & &\NT{real}           \\
&\oplus  & &\NT{binary}         &  &b       & &\NT{bool}           \\
&\ominus & &\NT{unary}          &  &c       & &\NT{character}      \\
\\
&\text{\underline{Sentencias}}
\\
&s       & &\NT{sentence}       &  &ss      & &\NT{sentences}      \\
\\
&\text{\underline{Tipos}}
\\
&\theta & &\NT{type}            &  &as      & &\NT{arraysize}      \\
&td     & &\NT{typedecl}        &  &tn      & &\NT{tname}          \\
&tv     & &\NT{typevariable}    &  &cn      & &\NT{cname}          \\
&cl     & &\NT{class}           &  &fn      & &\NT{fname}          \\
&io     & &\NT{io}              &  &fd      & &\NT{field}          \\
\\
&\text{\underline{Programas}}
\\
&fpd     & &\NT{funprocdecl}    &  &fa      & &\NT{funargument}    \\
&vd      & &\NT{variabledecl}   &  &fr      & &\NT{funreturn}      \\
&cs      & &\NT{constraints}    &  &pa      & &\NT{procargument}   \\
\end{align*}

\section{Notación}

% IDEA: Antes de comenzar propiamente con la definición de los distintos chequeos, es necesario describir el significado de la notación que emplearemos a lo largo del capítulo.
A continuación presentaremos algo de notación general sobre los diferentes \textit{juicios de tipado} que nos encontraremos en este capítulo.
Por lo general, diremos que un elemento sintáctico está \textit{bien tipado} cuando podamos construir una derivación de su respectivo juicio.
% IDEA: Cuando uno de los elementos verificados satisfaga todas las propiedades requeridas por el análisis, diremos que el mismo se encuentra \textit{bien tipado}.
Comúnmente, utilizaremos la siguiente notación para decir que la construcción sintáctica $\chi$ está \textit{bien tipada} bajo el contexto $\pi$, en base a las reglas de la categoría sintáctica $\gamma$.
\begin{gather*}
\pi \DASH{\gamma} \chi
\end{gather*}

% IDEA: También denominadas reglas para \textit{juicios de tipado}, estas construcciones a veces producirán resultados luego de finalizado su análisis.
En algunas ocasiones estos juicios prueban que la extensión de algún contexto determinado es válida.
De alguna manera podemos pensar que estos juicios tienen a tal extensión como resultado de su chequeo.
% IDEA: Los mismos podrán extender los contextos involucrados en la verificación, a medida que se recolecta información del programa, o también podrán generar nuevos elementos sintácticos, como es el caso para los chequeos de expresiones.
En estas situaciones se utilizará la siguiente notación, donde $\omega$ representa la extensión de alguno de los contextos involucrados.
\begin{gather*}
\pi \DASH{\gamma} \chi : \omega
\end{gather*}

% IDEA: Otra situación que se suele presentar a la hora del análisis, es el uso de una cantidad arbitraria de contextos.
Para facilitar la lectura muchas veces agrupamos (o por el contrario, desagrupamos) diferentes tipos de contextos.
Debido que comúnmente se deberá almacenar información de distintos índoles para efectuar la verificación, se hará uso de la siguiente notación para reflejar esta condición.
Notar que también existe la posibilidad que no se necesite utilizar ninguna información contextual, por lo que en estos escenarios se omite el listado de contextos.
\begin{gather*}
\pi_1, \ldots, \pi_n \DASH{\gamma} \chi : \omega
\end{gather*}

% IDEA: Cierto conjunto de reglas, utilizado para el chequeo de tipos, será empleado en más de una situación distinta a lo largo de la verificación del programa.
% IDEA: En cada una de estas ocurrencias, las propiedades que se desean validar, mediante la aplicación de las reglas, dependerán del entorno de análisis actual.
Ciertas construcciones sintácticas podrán ser consideradas \textit{bien tipadas} de acuerdo a determinados contextos locales $\overline{\pi}_{j}$, en estas situaciones se empleará la siguiente notación.
Los contextos mencionados representan información elemental necesaria para la derivación del juicio de tipado apropiado.
% IDEA: Por lo tanto, los conjuntos $\overline{\pi}_{j}$ representarán la información esencial que condicionará la aplicación de estas reglas.
\begin{gather*}
\pi_1, \ldots, \pi_n \DASH[\overline{\pi}_{1}, \ldots, \overline{\pi}_{m}]{\gamma} \chi : \omega
\end{gather*}

A lo largo de este capítulo, se darán distintos conjuntos de reglas $\gamma$ en base al elemento sintáctico que $\chi$ represente.
Al mismo tiempo, se definirán diversos contextos $\pi$ que almacenarán la información recopilada a lo largo de una, o más, derivaciones de distintos juicios.
% IDEA: Sumado a todo esto, la clase de resultados $\omega$ obtenidos durante la verificación también dependerá del entorno de análisis en el que nos encontremos inmersos.

% IDEA: Una última notación que puede ser conveniente presentar, es la utilizada para expandir contextos.
Los contextos que emplearemos a lo largo del capítulo serán conjuntos compuestos por \textit{n-uplas} de elementos sintácticos del lenguaje.
Debido que la operación principal sobre estas construcciones será el agregado de distintas componentes que conforman al elemento sintáctico recientemente analizado, se hará uso de la siguiente abreviatura para simplificar la notación.
\begin{gather*}
(\chi_1, \ldots, \chi_n) \triangleright \pi \overset{def}{=} \{ (\chi_1, \ldots, \chi_n) \} \cup \pi
\end{gather*}

\section{Contextos y Juicios}

% IDEA: Comenzando con la especificación de los chequeos, avanzaremos progresivamente en el análisis de un programa a medida que las distintas propiedades sean enunciadas y verificadas.
Para asegurar la corrección estática de un programa, se deben validar cada una de sus componentes, y en el caso que todas superen su respectiva verificación, se dirá que el mismo se encuentra \textit{bien tipado}.
Según la sintaxis del lenguaje, un programa posee la siguiente estructura, donde vale que $n \geq 0$ y $m > 0$; lo cual implica que un programa tiene al menos una función o un procedimiento declarado.
\begin{gather*}
typedecl_1
\\
\ldots
\\
typedecl_n
\\
funprocdecl_1
\\
\ldots
\\
funprocdecl_m
\end{gather*}